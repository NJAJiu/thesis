
% system_process.tex

\chapter{System Process}
\label{cha:system_process}

As mentioned in Section~\ref{sec:processes}, after peer/poll processes deal
with data from servers, the peer statistics are passed to the system process.
There are three algorithms in the system process:
\begin{itemize}
    \item Selection algorithm
    \item Clustering algorithm
    \item Mitigation algorithm
\end{itemize}

To understand the system process, we first introduce some concepts.

\section{Concepts}%
\label{sec:system_concepts}
\begin{enumerate}
    \item Truechimer and falseticker\\
        A truechimer is a clock that maintains timekeeping accuracy to a
        previously published (and trusted) standard, while a falseticker is a
        clock that does not.
    \item Root distance $\lambda$\\
        Root distance is a very important statistic used in system process, it
        ``represents the maximum error of the estimate due to all
        causes.''~\cite{performance_metrics} The root distance is calculated as
        the \emph{root delay} from the primary source of time plus the
        \emph{root dispersion} of that source. Here the source is the reference
        clock, the root delay and the root dispersion are similar with the
        statistics delay and dispersion we discussed in
        Chapter~\ref{cha:peer/poll_processes}, but relate to the source. We
        will discuss relative calculations in
        Section~\ref{sec:selection_algorithm}.
\end{enumerate}

(TODO: add purpose and figure of this part)

\section{Selection algorithm}%
\label{sec:selection_algorithm}
The selection algorithm takes peer statistics of each peer process then
determines truechimers among peers. The first thing is to calculate root
distance $\lambda$ for each peer.

\subsection{Root distance calculation}%
\label{sub:root_distance_calculation}

\begin{myverbbox}
    {\mindisp}MINDISP
\end{myverbbox}

As mentioned in Section~\ref{sec:system_concepts}, the root distance has two
parts: \emph{root delay} $\Delta$ and \emph{root dispersion} $E$. Furthermore, 
we have: 
\begin{equation}
    \Delta = \max(\mindisp, \delta_p + \delta)
    \label{eq:root_delay}
\end{equation}
and
\begin{equation}
    E = \varepsilon_p + \varepsilon + \varphi
    \label{eq:root_dispersion}
\end{equation}
Then we have root distance $\lambda$ which is calculated by:
\begin{align}
    \lambda & = \frac{\Delta}{2} + E\\
    & =\frac{\max(\mindisp, \delta_p + \delta)}{2} 
    + \varepsilon_p + \varepsilon + \varphi
    \label{eq:root_distance}
\end{align}
where 
\begin{itemize}
    \item $\mindisp$ is the minimum increment of dispersion.  It is used to
        avoid timing loops in NTP subnets with very fast processors and
        networks.~\cite{rfc5905} Its default value has three versions among
        official documents of NTP: 5 ms,~\cite{rfc5905} 10 ms,~\cite{rfc5905}
        and 1 ms.~\cite{performance_metrics}
    \item $\delta_p$ is the root delay of the peer, it is in the NTP packet
        header.
    \item $\varepsilon_p$ is the root dispersion of the peer, it is in the NTP
        packet header.
    \item $\delta$, $\varepsilon$ and $\varphi$  are the peer delay, peer
        dispersion and peer jitter we mentioned in
        Chapter~\ref{cha:peer/poll_processes}. Note that the peer dispersion
        $\varepsilon$ should be updated every time we access to it since it
        increases in a constant speed $\phi = 15 ppm = 15 \mu s/s$.
\end{itemize}

\subsection{Pre-selection checking}%
\label{sub:pre_selection_checking}
After the calculation of root distance for each peer, a number of checks are
performed, which includes the following checks:
\begin{enumerate}
    \item Stratum Error\\
        As we mentioned before, the stratum numbers are between 0 and 15
        inclusive. If the peer is never synchronized or its stratum number is
        not in the range, an stratum error occurs.
    \item Distance Error\\
        There is an option called \verb|MAXDIST| which by default is 1.5s for
        networks on the Earth and 2.5s for networks including the
        Moon.~\cite{clock_selection} (In the RFC file, its default value is
        1s~\cite{rfc5905}.)
    \item Loop Error\\
        If the peer is synchronized to the client, a loop error occurs.
    \item Unreachable Error\\
        If the server is Unreachable, an unreachable error occurs.
\end{enumerate}
If any one of the errors occurs, the relative peer is considered as
nonselectable and will not be considered in the following selection
algorithm.

\subsection{Background}%
\label{sub:selection_algorithm_background}

% correctness interval
For each peer, given the offset $\theta_i$ and the root distance $\lambda_i$,
we define a correctness interval of that peer: $[\theta_i - \lambda_i,
\theta_i + \lambda_i]$. By the definition of offset and root distance, if a
peer is a truechimer, we know that the true offset $\theta$ should be in the
interval, where the true offset is the offset of the reference clock relative
to the client.

In ideal situation that all peers are truechimers, and the reference clocks
that the peers synchronized to have very small amount of offsets among them,
the true offset should be in the intersection of all correctness intervals. In
practice, there may be some falsetickers, whose correctness intervals may have
two situations:
\begin{enumerate}
    \item The correctness interval of every falseticker has no intersection
        with the correctness intervals of all truechimers.\\ This is a good
        situation that we can determine falsetickers by correctness intervals.
    \item The correctness intervals of some falsetickers have intersection with
        the correctness intervals of all truechimers.\\ This is a bad situation
        that we cannot determine the falsetickers and the \emph{intersection
        interval} defined below will not contains the true offset.  However, we
        cannot do anything in this situation.
\end{enumerate}

Now we define the \emph{intersection interval} which is considered to contain
the true offset. The \emph{intersection interval} is the largest interval
which is subset of most correctness intervals. Note that the definition is
not the official one since it is considered to be confused.
Figure~\ref{fig:intersection_interval} shows an example of intersection
interval. There are four peers, which are also called candidates, A, B, C and
D. Their correctness intervals are the bars in the figure. As we can see,
there is no common intersection for all of them, but there is one for A, B
and C. In this case, candidate D is considered as a falseticker and the
intersection interval is indicated by the two dash lines which are the lower
and upper boundary of it.

% intersection_interval.tex


\begin{figure}[htpb]
\begin{center}
\begin{tikzpicture}[scale=0.7, transform shape,
        squarednode/.style={rectangle, draw=black, very thick, 
        align=center},
        border/.style={-, draw=black, very thick, dashed},
    ]
    % intervals
    \node[squarednode, minimum width=29mm] (b) {B};
    \node[squarednode, minimum width=35mm] (a) at ($(b.south) + (0.7, -1)$) {A};
    \node[squarednode, minimum width=50mm] (c) at ($(a.south) + (1.3, -1)$) {C};
    \node[squarednode, minimum width=20mm] (d) [left=15mm of c] {D};

    % boundary of intersection interval
    \coordinate (c1) at ( $(c.west) + (0, -1)$);
    \coordinate (b1) at ( $(b.west) + (0 , 1)$);
    \draw[border] (c1)node{} -- (c1 |- b1)node{};

    \coordinate (c2) at ( $(c.east) + (0, -1)$); 
    \coordinate (b2) at ( $(b.east) + (0 , 1)$);
    \draw[border] (b2)node{} -- (c2 -| b2)node{};

\end{tikzpicture}
\end{center}
    \caption{Intersection interval}
\label{fig:intersection_interval}
\end{figure}


% fig:intersection_interval

\subsection{Selection algorithm}%
\label{sub:selection_algorithm}
The selection algorithms is used to cleave truechimers from falsetickers. It
does this by calculating the intersection interval given correctness intervals
from all candidates and "was devised by Keith Marzullo in his doctoral
dissertation.  It was first implemented in the Digital Time Synchronization
Service (DTSS) for the VMS operating system for the
VAX\null."~\cite{clock_selection}

Algorithm~\ref{alg:intersection_interval} shows how to get the intersection
interval.  In the algorithm, $endpoints$ is a list of tuples where each tuple
represents an endpoint of a correctness interval, it contains the id of the
peer, an integer $type$ to indicate whether it is the high($1$) or low($-1$)
endpoint of the correctness interval, and the $value$ of the endpoint. It is
initially empty.

\begin{algorithm}[!t]
    \centering
    \small
    \caption{Calculates the intersection interval.}
    \begin{algorithmic}[1]
        \REQUIRE
            correctness intervals.
        \ENSURE
            calculates the low and high endpoints of intersection interval.
        \STATE
            \COMMENT{initialization}
        \STATE
            $n \leftarrow 0, low \leftarrow 1e9, high \leftarrow -1e9$
        \FOR{$i \leftarrow$ correctness interval}
            \STATE
            Adds two tuples to $endpoints$
            \STATE
            $n \leftarrow n + 1$
        \ENDFOR
        \STATE
            Sorts $endpoints$ by the value.
        \STATE
        \STATE
            \COMMENT{Start calculation}
        \FOR{$allow \leftarrow \left[0, n/2 \right]$}
            \STATE
                \COMMENT{low points}
            \STATE
            $count \leftarrow 0$
            \FOR{$i \leftarrow \left[0, 2n \right]$}
                \STATE
                $low \leftarrow endpoints[i].value$
                \STATE
                $count \leftarrow count - endpoints[i].type$
                \IF{$count \geq n - allow$}
                \STATE
                    $break$
                \ENDIF
            \ENDFOR
            \STATE
            \STATE
                \COMMENT{high points}
            \STATE
            $count \leftarrow 0$
            \FOR{$i \leftarrow \left[2n, 0 \right]$}
                \STATE
                $high \leftarrow endpoints[i].value$
                \STATE
                $count \leftarrow count + endpoints[i].type$
                \IF{$count \geq n - allow$}
                    \STATE
                    $break$
                \ENDIF
            \ENDFOR
            \STATE
            \STATE
                \COMMENT{Check whether the intersection interval is found.}
            \IF{$high > low$}
                \STATE
                $break$
            \ENDIF
        \ENDFOR
    \end{algorithmic}
\label{alg:intersection_interval}
\end{algorithm}

Note that there are some details that we need to discuss. 
\begin{enumerate}
    \item 
        The initial values of $low$ and $high$ are not used unless the
        algorithm is called when there is no candidates.
    \item 
        The selection sort algorithm is used here to sort the list
        $endpoints$. It is not efficiency in principle, but it is acceptable
        here since there should not be hundreds of thousands of tuples to be
        sorted. In practice, each peer is manually assigned by the user of
        NTP, so even the number of tuples should be as twice as the number of
        peers, it will not be several hundreds. (Maybe no more than twenty.)
    \item
        The variable $allow$ indicates the number of falsetickers. From the
        algorithm, we can see that if we cannot get an intersection interval
        which is the intersection of more than half correctness intervals,
        the algorithm gives up and no truechimers can be
        found.~\cite{clock_selection}
    \item
        The algorithm first try to find an intersection of all correctness
        intervals. If it fails, it increase $allow$ by one, then try to find
        an intersection interval for one less correctness intervals. Then it
        repeats the step until an intersection interval is found or $allow$
        is to large.
\end{enumerate}
If the algorithm successfully ends, the truechimers left are used in the
clustering algorithm, otherwise nothing further happens.  The system clock of
the client will not be synchronized.

Note that the algorithm ``implementation'' in RFC5905 page 91 is completely
wrong.

\subsection{Midpoints of correctness intervals}
\label{midpoints_of_correctness_intervals}
(NOTE: finish this part after finishing this chapter)

\section{Clustering algorithm}%
\label{sec:clustering_algorithm}
For truechimers, we can  calculate the \emph{selection jitter} statistic for
each of them, where it is the root-mean-square (RMS) of offset differences
between the peer and all others. The selection jitter of peer $i$ is denoted by
$\varphi_{s,i}$ and the calculation is:
\begin{equation}
    \varphi_{s,i} = \sqrt{\frac{1}{m-1} \sum^{m-1}_{j=0} (\theta_i -
    \theta_j)^2 }
    \label{eq:selection_jitter}
\end{equation}
where $m$ is the number of peers left.

The clustering algorithm does the following things:
\begin{enumerate}
    \item Calculates selection jitters $\varphi_{s, i}$ for each peer $i$.
    \item Do the terminating tests, which contains two tests:
        \begin{enumerate}
            \item If $m < \verb|MINCLOCK|$, 
                stops the algorithm, where $m$ is
                the number of peers left and \verb|MINCLOCK| is a parameter
                indicates the minimum number of survivors in this algorithm
                whose default value is 3.~\cite{rfc5905}
            \item If $\max(\varphi_{s,i}) < \min(\varphi_i)$, stops the
                algorithm, where $\varphi_i$ is the jitter of peer $i$.
        \end{enumerate}
    \item Eliminates the peer with largest selection jitter, decreases $m$
        by one, repeats step 1.
\end{enumerate}
The first one of terminating tests is quite straight forward, we add some
detail about the second one. Recall that the peer jitter $\varphi$ is the
root-mean-square (RMS) of offset differences between selection samples and
valid ones in the shift register, as we mentioned in
Section~\ref{sub:clock_filter_algorithm}. It indicates how stable the peer is.
The selection jitter indicates how ``far'' a truechimer's offset is away from
others.  So if $\max(\varphi_{s,i}) > \min(\varphi_i)$, that means there is at
least one truechimer's offset which is relatively far from others. Otherwise,
all of them are close enough.

After applying the clustering algorithm, there are some peers left, which are
called survivors. We sort
them by their root distances $\lambda$. The current maximum selection jitter
is saved as the system selection jitter $\varphi_s$. Note that official documents
of NTP mentioned that we should sort the survivors before applying clustering
algorithm, which is slightly strange.

\section{Combine algorithm}%
\label{sec:combine_algorithm}
The combine algorithm is used to compute two system statistics: system offset 
$\Theta$ and system jitter $\vartheta$. Let $\theta_i, \varphi_i, \lambda_i$
represent the peer offset, peer jitter and the root distance of the $i$th
survivor. Here we do the following calculations:~\cite{redbook}
\begin{equation}
    \Theta = a \sum^{}_{i} \frac{\theta_i}{\lambda_i} \text{ and } 
    \varphi_r = \sqrt{a \sum^{}_{i} \frac{\varphi_i ^ 2}{\lambda_i}}
    \label{eq:system_offset_selection_jitter}
\end{equation}
where $a$ is the normalizer
\begin{equation}
    a = \left( \sum^{}_{i} \frac{1}{\lambda_i} \right) ^ {-1}
    \label{eq:normalizer}
\end{equation}
and $\varphi_r$ is called the \emph{system peer jitter}.~\cite{rfc5905}

Then we calculate the system jitter:~\cite{redbook}
\begin{equation}
    \vartheta = \sqrt{\varphi_s^2 + \varphi_r^2}
    \label{eq:system_jitter}
\end{equation}

About the calculations:
\begin{enumerate}
    \item The system offset is the weighted average of peer offsets. The
        weights are the reciprocals of root distances. This means that peers
        have less root distance contributes more to the system offset. As
        mentioned in Section~\ref{sec:system_concepts}, the root distance
        represents the maximum error due to all causes. We prefer peers with
        low errors as well as it show in the calculation.
    \item The system peer jitter is the root-weighted-average-square of peer
        jitters, where the weights are as the same as the weights in the
        calculation of system offset.
    \item The system jitter is the geometric average of selection jitter and
        system peer jitter.
\end{enumerate}

\section{System peer selection}%
\label{sec:system_peer_selection}


