
% system_process.tex

\chapter{System Process}
\label{cha:system_process}

As mentioned in Section~\ref{sec:processes}, after the peer/poll processes deal
with the data from servers, the peer statistics are passed to the system
process.  There are three algorithms in the system process:
\begin{itemize}
    \item Selection algorithm
    \item Clustering algorithm
    \item Combining algorithm
\end{itemize}

\section{Definitions}%
\label{sec:system_concepts}
To understand the system process, we first introduce some definitions.

\begin{enumerate}
    \item Truechimer and falseticker\\
        A \emph{truechimer} is a clock that maintains timekeeping accuracy to a
        previously published (and trusted) standard, while a \emph{falseticker}
        is a clock that does not~\cite{redbook}.
    \item Root distance $\lambda$\\
        \emph{Root distance} is a very important statistic used in system
        process, it ``represents the maximum error of the estimate due to all
        causes''~\cite{performance_metrics}. The root distance of a peer is
        calculated as the \emph{root delay} from the peer's primary source of
        time plus the \emph{root dispersion} of that source. Here the source is
        the reference clock to which the peer is synchronized; the root delay
        is the total round-trip delay to the reference clock and the root
        dispersion is the total dispersion to the reference clock. The root
        delay and the root dispersion are similar to the statistics delay and
        dispersion we discussed in Chapter~\ref{cha:peer/poll_processes}, but
        relate to the source. We will discuss relative calculations in
        Section~\ref{sec:selection_algorithm}.
\end{enumerate}

\section{Overview}%
\label{sec:system_overview}

Figure~\ref{fig:system_process} shows the basic structure of system process.
The selection algorithm takes peer statistics calculated by peer processes,
then eliminates falsetickers. The clustering algorithm selects the best
truechimers then the combining algorithm calculates system statistics and pass
them to the clock discipline algorithm.

% figurs/system_process.tex

\begin{figure}[htpb]
    \begin{center}
    \begin{tikzpicture}[scale=0.7, transform shape,
            squarednode/.style={rectangle, draw=black, very thick, minimum
            size=15mm, minimum width=30mm, align=center},
            squarednode1/.style 2 args={draw=black, very thick, minimum
            size=15mm,fit={(#1.north) (#2.south)}, text width=, inner sep=2mm, text
            centered, align=center},
            circlenode/.style={circle, draw=black, very thick, minimum size=15mm},
            border/.style={-, draw=black, very thick, dashed},
        ]
    % Nodes

    % system process
    \node[squarednode] (sa1) {Selection\\ algorithm};
    \node[squarednode] (sa2) [right=of sa1] {Clustering\\ algorithm};
    \node[squarednode] (sa3) [right=of sa2] {Combining\\ Algorithms};
    \node[align=center](lab3)[above=5mm of sa2] {System\\ Process};
    \draw[border] ($(lab3.north east) + (1.0, 0.2)$) 
        -| ($(sa1.south west) + (-0.4, -0.8)$) 
        -- ($(sa3.south east) + (0.4, -0.8)$) 
        |- ($(lab3.north east) + (1.0, 0.2)$);

    % peer
    \node[squarednode] (p) [left=of sa1] {Peer\\ statistics};

    % clock discipline
    \node[squarednode] (cd) [right=of sa3] {Clock discipline\\ process};
    % Lines

    % system process
    \draw[-latex, thick] (sa1.east) -- (sa2.west);
    \draw[-latex, thick] (sa2.east) -- (sa3.west);
    \draw[-latex, thick] (p.east) -- (sa1.west);
    \draw[-latex, thick] (sa3.east) -- (cd.west); 
        
    
\end{tikzpicture}
\end{center}
\caption{System process overview}
\label{fig:system_process}
\end{figure}
    
    
    

% fig:system_process

\section{Selection algorithm}%
\label{sec:selection_algorithm}
The selection algorithm takes peer statistics of each peer process then
determines truechimers among peers. The first thing is to calculate root
distance $\lambda$ for each peer.

\subsection{Root distance calculation}%
\label{sub:root_distance_calculation}

\begin{myverbbox}
    {\mindisp}MINDISP
\end{myverbbox}

As mentioned in Section~\ref{sec:system_concepts}, the root distance has two
parts: \emph{root delay} $\Delta$ and \emph{root dispersion} $E$. Furthermore, 
we have: 
\begin{equation}
    \Delta = \max(\mindisp, \delta_p + \delta)
    \label{eq:root_delay}
\end{equation}
and
\begin{equation}
    E = \varepsilon_p + \varepsilon + \varphi
    \label{eq:root_dispersion}
\end{equation}
where 
\begin{itemize}
    \item $\mindisp$ is the minimum increment of dispersion. It is used to
        avoid timing loops in NTP subnets with very fast processors and
        networks~\cite{rfc5905}. Its default value has three versions among
        official documents of NTP: 5 ms~\cite{rfc5905}, 10 ms~\cite{rfc5905},
        and 1 ms~\cite{performance_metrics}.
    \item $\delta_p$ is the root delay of the peer, it is in the NTP packet
        header.
    \item $\varepsilon_p$ is the root dispersion of the peer, it is in the NTP
        packet header.
    \item $\delta$, $\varepsilon$ and $\varphi$ are the peer delay, peer
        dispersion and peer jitter we mentioned in
        Chapter~\ref{cha:peer/poll_processes}. Note that the peer dispersion
        $\varepsilon$ should be updated every time we access it since it
        increases in a constant speed $\phi = 15 ppm = 15 \mu s/s$.
\end{itemize}
Then we have the root distance $\lambda$ which is calculated as follows:
\begin{align}
    \lambda & = \frac{\Delta}{2} + E\\
    & =\frac{\max(\mindisp\, \delta_p + \delta)}{2} 
    + \varepsilon_p + \varepsilon + \varphi
    \label{eq:root_distance}
\end{align}
Equation~\ref{eq:root_distance} shows the two parts of the root distance, where
half of the root delay is the error cased by asymmetric network delays and the
root dispersion is the error caused by precisions of servers and the client.

\subsection{Pre-selection checking}%
\label{sub:pre_selection_checking}
After the calculation of root distance for each peer, a number of checks are
performed, which includes the following:
\begin{enumerate}
    \item Stratum Error\\
        As we mentioned before, the stratum numbers are between 0 and 15
        inclusive. If the peer is never synchronized or its stratum number is
        not in this range, a stratum error occurs.
    \item Distance Error\\
        There is an option called \verb|MAXDIST| which by default is 1.5s for
        networks on the Earth and 2.5s for networks including the
        Moon~\cite{clock_selection}. (In RFC 5905, its default value is
        1s~\cite{rfc5905}.)
    \item Loop Error\\
        If the peer is synchronized to the client, a loop error occurs.
    \item Unreachable Error\\
        If the server is unreachable, an unreachable error occurs.
\end{enumerate}
If any one of these errors occurs, the relevant peer is considered as
nonselectable and will not be considered in the following selection
algorithm.

\subsection{Background}%
\label{sub:selection_algorithm_background}

% correctness interval
For each peer, given the offset $\theta_i$ and the root distance $\lambda_i$,
we define a \emph{correctness interval} of that peer: $[\theta_i - \lambda_i,
\theta_i + \lambda_i]$. The correctness interval is called the \emph{confidence
interval} in~\cite{redbook}. By the definition of offset and root distance, if
a peer is a truechimer, we know that the true offset $\theta$ should be in the
interval, where the true offset is the offset of the reference clock relative
to the client.

In the ideal situation where all peers are truechimers, and the reference clocks
to which the peers are synchronized have very small amounts of offset among them,
the true offset should be in the intersection of all the correctness intervals. In
practice, there may be some falsetickers, whose correctness intervals may fall
into one of 
two situations:
\begin{enumerate}
    \item The correctness interval of every falseticker has no intersection
        with the correctness intervals of all truechimers.\\ 
        This is a good situation, in that we can determine falsetickers by
        comparing the correctness intervals.
    \item The correctness intervals of some falsetickers have a non-empty
        intersection with the correctness intervals of all truechimers.\\ 
        This is a bad situation, in that we cannot determine the falsetickers and
        the \emph{intersection interval} defined below may not contains the
        true offset. However, in this case, the falsetickers can not be
        determined and the result of selection algorithm will be wrong.
\end{enumerate}

We now define the \emph{intersection interval}, which is considered to contain
the true offset. The \emph{intersection interval} is the largest interval
which is a subset of the largest number of correctness intervals. Note that this
definition is slightly modified from the official one, which is ``the smallest
interval containing points from the largest number of correctness
intervals''\cite{clock_selection}. The official definition is confused since the
``smallest'' interval may be a interval with just one point in the interval of
our new definition.
Figure~\ref{fig:intersection_interval} shows an example of an intersection
interval. There are four peers, which are also called candidates A, B, C and
D\null. Their correctness intervals are the bars in the figure. As we can see,
there is no common intersection for all of them, but there is one for A, B
and C\null. In this case, peer D is considered as a falseticker and the
intersection interval is indicated by the two dashed lines which are the lower
and upper boundaries of the intersection interval.

% intersection_interval.tex


\begin{figure}[htpb]
\begin{center}
\begin{tikzpicture}[scale=0.7, transform shape,
        squarednode/.style={rectangle, draw=black, very thick, 
        align=center},
        border/.style={-, draw=black, very thick, dashed},
    ]
    % intervals
    \node[squarednode, minimum width=29mm] (b) {B};
    \node[squarednode, minimum width=35mm] (a) at ($(b.south) + (0.7, -1)$) {A};
    \node[squarednode, minimum width=50mm] (c) at ($(a.south) + (1.3, -1)$) {C};
    \node[squarednode, minimum width=20mm] (d) [left=15mm of c] {D};

    % boundary of intersection interval
    \coordinate (c1) at ( $(c.west) + (0, -1)$);
    \coordinate (b1) at ( $(b.west) + (0 , 1)$);
    \draw[border] (c1)node{} -- (c1 |- b1)node{};

    \coordinate (c2) at ( $(c.east) + (0, -1)$); 
    \coordinate (b2) at ( $(b.east) + (0 , 1)$);
    \draw[border] (b2)node{} -- (c2 -| b2)node{};

\end{tikzpicture}
\end{center}
    \caption{Intersection interval~\cite{clock_selection}}
\label{fig:intersection_interval}
\end{figure}


% fig:intersection_interval

There is another problem with intersection interval. The offset of peers are
the midpoints of correctness intervals. It is said that the intersection
interval should contains all midpoints of correctness intervals of
truchimers~\cite{redbook}. But we did not define the intersection interval in
this way because in the source code of the NTP~\cite{source_code}, the
intersection interval is determined without considering the midpoints of
correctness intervals.

\subsection{Selection algorithm}%
\label{sub:selection_algorithm}
The selection algorithm is used to cleave truechimers from falsetickers. It
does this by calculating the intersection interval given correctness intervals
from all candidates, the algorithm ``was devised by Keith Marzullo in his
doctoral dissertation.  It was first implemented in the Digital Time
Synchronization Service (DTSS) for the VMS operating system for the
VAX\null''~\cite{clock_selection}.

Algorithm~\ref{alg:intersection_interval}~\cite{source_code} shows how to
calculate the intersection interval.  In the algorithm, $endpoints$ is a list
of tuples, where each tuple represents an endpoint of a correctness interval.
Each tuple is comprised of the id of the peer, an integer $type$ to indicate
whether it is the high ($1$) or low ($-1$) endpoint of the correctness
interval, and the $value$ of the endpoint. It is initially empty.

\begin{algorithm}[!t]
    \centering
    \small
    \caption{Calculates the intersection interval.}
    \begin{algorithmic}[1]
        \REQUIRE
            correctness intervals.
        \ENSURE
            calculates the low and high endpoints of intersection interval.
        \STATE
            \COMMENT{initialization}
        \STATE
            $n \leftarrow 0, low \leftarrow 1e9, high \leftarrow -1e9$
        \FOR{$i \leftarrow$ correctness interval}
            \STATE
            Adds two tuples to $endpoints$
            \STATE
            $n \leftarrow n + 1$
        \ENDFOR
        \STATE
            Sorts $endpoints$ by the value.
        \STATE
        \STATE
            \COMMENT{Start calculation}
        \FOR{$allow \leftarrow \left[0, n/2 \right]$}
            \STATE
                \COMMENT{low points}
            \STATE
            $count \leftarrow 0$
            \FOR{$i \leftarrow \left[0, 2n \right]$}
                \STATE
                $low \leftarrow endpoints[i].value$
                \STATE
                $count \leftarrow count - endpoints[i].type$
                \IF{$count \geq n - allow$}
                \STATE
                    $break$
                \ENDIF
            \ENDFOR
            \STATE
            \STATE
                \COMMENT{high points}
            \STATE
            $count \leftarrow 0$
            \FOR{$i \leftarrow \left[2n, 0 \right]$}
                \STATE
                $high \leftarrow endpoints[i].value$
                \STATE
                $count \leftarrow count + endpoints[i].type$
                \IF{$count \geq n - allow$}
                    \STATE
                    $break$
                \ENDIF
            \ENDFOR
            \STATE
            \STATE
                \COMMENT{Check whether the intersection interval is found.}
            \IF{$high > low$}
                \STATE
                $break$
            \ENDIF
        \ENDFOR
    \end{algorithmic}
\label{alg:intersection_interval}
\end{algorithm}

Note that there are some details that we need to discuss. 
\begin{enumerate}
    \item 
        The initial values of $low$ and $high$ are not used unless the
        algorithm is called when there is no candidates.
    \item 
        The selection sort algorithm is used here to sort the list
        $endpoints$~\cite{source_code}. Selection sort is not efficient in
        general, but
        it is acceptable here since there should not be hundreds of thousands
        of tuples to be sorted. In practice, each peer is manually assigned by
        the NTP user, so even if the number of tuples is twice the
        number of peers, it will not be several hundreds, (maybe no more than
        twenty). It is also interesting to use selection sort in the selection
        algorithm.
    \item
        The variable $allow$ indicates the number of falsetickers. From the
        algorithm, we can see that if we cannot get an intersection interval
        which is the intersection of more than half correctness intervals,
        the algorithm gives up and no truechimers can be
        found~\cite{clock_selection}.
    \item
        The algorithm first tries to find an intersection of all correctness
        intervals. If it fails, it increases $allow$ by one, then tries to find
        an intersection interval of one fewer correctness intervals. The
        algorithm
        repeats this step until an intersection interval is found or $allow$
        is too large.
\end{enumerate}
If the algorithm ends successfully, the remaining truechimers are used in the
clustering algorithm, otherwise nothing further happens: the system clock of
the client will not be synchronized.

Note that the algorithm ``implementation'' in RFC 5905 (page 91) is 
wrong~\cite{rfc5905}. In that code, the number of peers is incorrectly
calculated as the number of end points and midpoints of correctness intervals,
which is three times the correct number.

% \subsection{Midpoints of correctness intervals}
% \label{midpoints_of_correctness_intervals}
% (NOTE: finish this part after finishing this chapter)

\section{Clustering algorithm}%
\label{sec:clustering_algorithm}
For each truechimers, we can calculate the \emph{selection jitter} statistic,
this is the root-mean-square (RMS) of offset differences between the truechimer
and all others. The selection jitter of peer $i$ is denoted by $\varphi_{s,i}$
and the calculation is:
\begin{equation}
    \varphi_{s,i} = \sqrt{\frac{1}{m-1} \sum^{m-1}_{j=0} (\theta_i -
    \theta_j)^2 }
    \label{eq:selection_jitter}
\end{equation}
where $m$ is the number of truchimers.

The clustering algorithm does the following things:
\begin{enumerate}
    \item Calculates selection jitters $\varphi_{s, i}$ for each peer $i$.
    \item Performs the following two terminating tests:
        \begin{enumerate}
            \item If $m < \verb|MINCLOCK|$, stops the algorithm, where $m$ is
                the number of peers left and \verb|MINCLOCK| is a parameter
                indicating the minimum number of survivors in this algorithm,
                whose default value is 3~\cite{rfc5905}.
            \item If $\max(\varphi_{s,i}) < \min(\varphi_i)$, where
                $\varphi_i$ is the jitter of peer $i$, stops the
                algorithm.
        \end{enumerate}
    \item Eliminates the peer with largest selection jitter, decreases $m$
        by one, and repeats step 1 and 2.
\end{enumerate}
The first one of the terminating tests is quite straightforward, but we add some
detail about the second one. Recall that the peer jitter $\varphi$ is the
root-mean-square (RMS) of offset differences between selection samples and
valid ones (non-dummy tuples) in the shift register, as mentioned in
Section~\ref{sub:clock_filter_algorithm}. It indicates the variance of sample
offsets of the same peer which means how ``far'' sample offsets are from each
other.
The selection jitter of a truchimer represents how ``far'' its offset is away from
others.  So if $\max(\varphi_{s,i}) \ge \min(\varphi_i)$, that means there is at
least one truechimer's offset which is relatively far from the others. Otherwise,
all of them are close enough to each other.

After applying the clustering algorithm, there are some peers left, which are
called survivors. We sort them by a merit factor $\lambda_p = stratum_p \times
\verb|MAXDIST| + \lambda$, from low to high, where $stratum_p$ is the stratum
number of the peer. The selection jitter of the first peer of the sorted
survivors list is saved as the system selection jitter $\varphi_s$. Note that
the NTP official documents~\cite{redbook}~\cite{rfc5905} mentioned that we
should sort the survivors before applying the clustering algorithm, which is
slightly strange. In~\cite{rfc5905}, it is said that the system selection
jitter should be the maximum selection jitter of all survivors, which is wrong
based on~\cite{redbook} and~\cite{source_code}.

\section{Combining algorithm}%
\label{sec:combine_algorithm}
The combining algorithm is used to compute two system statistics: the system
offset $\Theta$ and the system jitter $\vartheta$. The system offset is used
in the clock discipline process to discipline the system clock. The system
jitter is ``the expected error, represents the nondeterministic uncertainty of
the offset as estimated from the jitter contributions''~\cite{redbook}.
Let $\theta_i$, $\varphi_i$,
$\lambda_i$ represent the peer offset, peer jitter and the root distance of the
$i$th survivor. Here we do the following calculations:~\cite{redbook}
\begin{equation}
    \Theta = a \sum^{}_{i} \frac{\theta_i}{\lambda_i} \text{ and } 
    \varphi_r = \sqrt{a \sum^{}_{i} \frac{\varphi_i ^ 2}{\lambda_i}}
    \label{eq:system_offset_selection_jitter}
\end{equation}
where $a$ is the normalizer
\begin{equation}
    a = \left( \sum^{}_{i} \frac{1}{\lambda_i} \right) ^ {-1}
    \label{eq:normalizer}
\end{equation}
and $\varphi_r$ is called the \emph{system peer jitter}~\cite{rfc5905}.

Then we calculate the system jitter~\cite{redbook}:
\begin{equation}
    \vartheta = \sqrt{\varphi_s^2 + \varphi_r^2}
    \label{eq:system_jitter}
\end{equation}

About the calculations:
\begin{enumerate}
    \item The system offset is the weighted average of the peer offsets. The
        weights are the reciprocals of the root distances. This means that
        peers have less root distance contribute more to the system offset. As
        mentioned in Section~\ref{sec:system_concepts}, the root distance
        represents the maximum error due to all causes.  Intuitively, peers
        with low errors seem preferable to peers with high errors, and the
        calculations reflect this preference.
    \item The system peer jitter is the root-weighted-average-square of peer
        jitters, where the weights are as the same as the weights in the
        calculation of system offset.
    \item The system jitter is the root-mean-square of the system peer jitter
        and the selection jitter. Note that the selection jitter is a part of
        the system peer jitter. From the calculation, we know that the system
        jitter is in fact the root-weighted-average-square of peer jitters,
        where the weight of the selection jitter is more than the sum of the
        other weights.
\end{enumerate}
Note the difference between the root distance and the system jitter. The root
distance is the maximum error, while the system jitter is the estimated error.
In practice, the error cannot exceed the root distance. ``This is an important
distinction because different applications may need one or the other, or both
of these statistics''~\cite{redbook}.

\section{System peer selection}%
\label{sec:system_peer_selection}
Users can designate one or more sources as \emph{prefer peer} but it is
usually not useful to designate more than one. Unless the prefer peer is
discarded by the selection algorithm, it becomes the \emph{system peer} and
the combining algorithm is not used. The system jitter and offset are the
prefer peer's jitter and offset respectively. Otherwise, the first peer in
the sorted list of survivors becomes the system peer. 

Some system variables come from the system peer. The important ones
include the stratum number, which is the stratum of the system peer plus one.
The system root delay and root dispersion are calculated using
Equation~\ref{eq:root_delay} and~\ref{eq:root_dispersion}.

In the next chapter, we will discuss how to discipline the system clock.
