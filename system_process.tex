
% system_process.tex

\chapter{System Process}
\label{cha:system_process}

As mentioned in Section~\ref{sec:processes}, after peer/poll processes deal
with data from servers, the peer statistics are passed to the system process.
There are three algorithms in the system process:
\begin{itemize}
    \item Selection algorithm
    \item Clustering algorithm
    \item Mitigation algorithm
\end{itemize}

To understand the system process, we first introduce some concepts.

\section{Concepts}%
\label{sec:system_concepts}
\begin{enumerate}
    \item Truechimer and falseticker\\
        A truechimer is a clock that maintains timekeeping accuracy to a
        previously published (and trusted) standard, while a falseticker is a
        clock that does not.
    \item Root distance $\lambda$\\
        Root distance is a very important statistic used in system process, it
        ``represents the maximum error of the estimate due to all
        causes.''~\cite{performance_metrics} The root distance is calculated as
        the \emph{root delay} from the primary source of time plus the
        \emph{root dispersion} of that source. Here the source is the reference
        clock, the root delay and the root dispersion are similar with the
        statistics delay and dispersion we discussed in
        Chapter~\ref{cha:peer/poll_processes}, but relate to the source. We
        will discuss relative calculations in
        Section~\ref{sec:selection_algorithm}.
\end{enumerate}

(TODO: add purpose and figure of this part)

\section{Selection algorithm}%
\label{sec:selection_algorithm}
The selection algorithm takes peer statistics of each peer process then
determines truechimers among peers. The first thing is to calculate root
distance $\lambda$ for each peer.

\subsection{Root distance calculation}%
\label{sub:root_distance_calculation}
As mentioned in Section~\ref{sec:system_concepts}, the root distance has two
parts: \emph{root delay} and \emph{root dispersion}. It is calculated by:
\begin{myverbbox}
    {\mindisp}MINDISP
\end{myverbbox}
\begin{equation}
    \lambda = \frac{\max(\mindisp, \delta_p + \delta)}{2} 
    + \varepsilon_p + \varepsilon + \varphi
    \label{eq:root_distance}
\end{equation}
where 
\begin{itemize}
    \item $\mindisp$ is the minimum increment of dispersion.  It is used to
        avoid timing loops in NTP subnets with very fast processors and
        networks.~\cite{rfc5905} Its default values have three versions among
        official documents of NTP: 5 ms,~\cite{rfc5905} 10 ms,~\cite{rfc5905}
        and 1 ms.~\cite{performance_metrics}
    \item $\delta_p$ is the root delay of the peer, it is in the NTP packet
        header.
    \item $\varepsilon_p$ is the root dispersion of the peer, it is in the NTP
        packet header.
    \item $\delta$, $\varepsilon$ and $\varphi$  are the peer delay, peer
        dispersion and peer jitter we mentioned in
        Chapter~\ref{cha:peer/poll_processes}. Note that the peer dispersion
        $\varepsilon$ should be updated every time we access to it since it
        increases in a constant speed $\phi = 15 ppm = 15 \mu s/s$.
\end{itemize}



\section{Clustering algorithm}%
\label{sec:clustering_algorithm}

\section{Mitigation algorithm}%
\label{sec:mitigation_algorithm}



