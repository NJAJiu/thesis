% conclusion.tex

\chapter{Conclusion}
\label{cha:conclusion}

As described earlier, the Network Time Protocol (NTP) was developed due to the
requirement of making a computer's system time accurate. The most commonly used
standard time is Coordinated Universal Time (UTC), which NTP uses as well.
UTC is kept by some reference clocks and only specific servers can access and
synchronize to these high-accurate clocks. These servers are called primary
servers. Primary servers
provide synchronization to others which means other devices can synchronize
their system times to primary servers. An end user, for example, 
running NTP daemon on her own laptop, should not synchronize to
primary servers, since in that case primary servers will be too busy. To deal
with this problem, secondary servers and the ``stratum'' concept were
introduced.
Secondary servers synchronize to other servers and provide synchronization but
they do not access to the reference clocks directly.   All primary servers are
in stratum one. The stratum number of
secondary server is equal to the stratum number of the server to that it synchronizes
to plus one. For example, a secondary server which synchronizes to a primary
server is in stratum two. 

Clients send synchronization request to servers, then adjust their
system clocks based on the feedback. The peer/poll processes deal with the
communication part. Poll processes send request to servers at poll intervals.
When a response packet arrives, offset and round-trip delay are calculated
based on the timestamps in the packet. We need four timestamps to do this job,
which are when the request packet is sent and received and when the response
packet is sent and received. Several recent results are recorded to measure
the reliability of servers. One problem with this procedure is the asymmetric
network delay.  Errors caused by this are almost impossible to properly handle.
An algorithm known as huff-n'-puff filter is used to attenuate these errors.

The system process takes results of the peer processes and produces a best
estimate of offset to synchronize the system clock. The selection algorithm
first looks at all peers and tries to eliminates falsetickers, which are peers
that are considered not to have an accurate time. The algorithm lets every peer
provide a correctness interval which is the maximum range that the accurate
time should be in. If most intervals have a common intersection, the rest which
do not have any point in the intersection are considered as falsetickers. 

The clustering algorithm tries to find a group of peers which have better
performance to synchronize the system clock. Although all peers left at that
time can provide accurate time, some of them may have a very large correctness
interval. In that case, the algorithm is not confident about the accurate time.
The clustering algorithm tries to eliminates some peers so that the remaining
correctness intervals are close to each other.

The combining algorithm computes system statistics used to discipline the system
clock in the clock discipline algorithm. If the algorithm runs on a server, the
results can be used in the header of response packets. Users can designate a
prefer peer (usually just one) so that the clustering and combining algorithms
can be skipped if the peer passed the selection algorithm and its peer
statistics can be used directly to discipline the system clock.

The clock discipline process and clock adjust process actually works with the
system clock. The clock discipline algorithm disciplines the time and frequency
of the system clock based on the result of the system process. However, it has some
mechanisms to avoid network spike and to reduce training time to get the intrinsic
frequency offset. NTP tries not to step the system time if possible. In that
case, the clock adjust process is used to slew it.

The clock discipline process also manage the poll interval of poll processes.
So the overview NTP architecture can be considered as a loop.

During the investigation, multiple errors and conflicts of the official
documents were found. The discussion in this thesis corrects apparent errors in
the official documents, and conflicts between documents are resolved by
examining the distributed NTP software~\cite{source_code}. 
Finally, this thesis added explanations above and beyond those found in the
official documentation in the hope that readers will be able to use this
document to better understand the operation of NTP.

