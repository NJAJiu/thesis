% conclusion.tex

\chapter{Conclusion}
\label{cha:conclusion}

As we discussed before, the Network Time Protocol (NTP) comes from the
requirement of making system time accurate. The most commonly used standard
time is Coordinated Universal Time (UTC), which NTP uses this as well. The UTC
is kept by some reference clocks and only some specific servers can access and
synchronize to them. The servers are called primary servers. Primary servers
provide synchronization to others which means other devices can synchronize
their system times to primary servers'. As a end user, for example, someone is
running NTP daemon on her own laptop, should not synchronize to primary servers
since in that case primary servers will be too busy. To deal with this problem,
we introduced secondary servers and the concept ``stratum''. Secondary servers
provide synchronization but they do not access to reference clocks directly.
Stratum is another way of saying ``layer''. All primary servers are in stratum
one. The stratum number of secondary server is equal to the stratum number of
server that it synchronizes to plus one. For example, a secondary server which
synchronizes to a primary server is in stratum two. 

As a client, we need to send synchronization request to servers, then adjust
system clock based on the feedback. The peer/poll processes deal with the
communication part. Poll processes send request to servers at poll intervals.
When a response packet arrives, we can calculate offset and round-trip delay
based on the timestamps in the packet. We need four timestamps to do this job,
which are when the request packet is sent and received and when the response
packet is sent and received. We also record several recent result to measure
the reliability of servers. One problem here is the asymmetric network delay.
Errors caused by this are almost impossible to fix. We use huff-n'-puff
filter to attenuate these errors.

System process takes results of peer processes, produce a best estimate of
offset to synchronize the system clock. The selection algorithm first looks at
all peers and tries to eliminates falsetickers, which are peers that are
considered not to have an accurate time. The algorithm let every peer provide a
correctness interval which is the maximum range that the accurate time should
be in. If most intervals have a common intersection, the rest which do not have
any point in the intersection are considered as falsetickers. 

The clustering tries to find a group of peers which have better performance to
synchronize system clock. Although all peers left at that time can provide
accurate time, some of them may have very large correctness interval. In that
case, they are not confident about where is the accurate time. The clustering
algorithm tries to eliminates some peers so that the left correctness intervals
are close to each other.

The combine algorithm computes system statistics used to discipline the system
clock in clock discipline algorithm. If the algorithm runs on a server, the
results can be used in the header of response packets. Users can designate
prefer peer (usually one) so that the clustering and combining algorithm can be
skipped if the peer passed the selection algorithm and its peer statistics can
be used directly to discipline the system clock.

The clock discipline process and clock adjust process actually works with
system clock. The clock discipline algorithm discipline the time and frequency
of system clock based on the result of system process. However, it has some
mechanisms to avoid network spike and to reduce training time to get intrinsic
frequency offset. NTP tries not to step the system time if possible. In that
case, clock adjust process is used to slew it.

The clock discipline process also manage the poll interval of poll processes.
So the overview NTP architecture can be considered as a loop.

During the investigation, some errors and conflicts of official documents are
found. Apparent errors are directly fixed while conflicts are changed based on
the implementation of NTP, which is~\cite{source_code}. There are some
explanation added to help us understand stuff such as the meaning of
statistics. 

