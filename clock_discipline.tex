
% clock_discipline.tex

\chapter{Clock discipline and clock adjust process}%
\label{cha:clock_discipline_process}
The next step of the process is to change the system clock of the client. As
mentioned in Section~\ref{sec:adjusting_system_time}, we want the system time
to be changed ``smoothly'', which means that we want to change it a only little
bit every time it is changed, but repeat such a change many times over an
acceptable period of time. This is a ``smooth'' way of changing the system
time. There are two thresholds
used in NTP: \emph{step} and \emph{panic}. When the offset exceeds the panic
threshold, which is 1000 seconds by default, we do not change the time. If the
offset is less than the panic threshold but more than the step threshold, which
is 128 ms by default, we change the time in the ``non-smooth'' way, which means
we directly set the system time to another value. In this case we say that
we the time is ``stepped''. If the offset is less than the step threshold, we
change the time ``smoothly'' as described before; in this case we say that the
time is ``slewed'' or ``adjusted''.

There is an illustrative example described in \cite{redbook}, which compares
adjusting the system clock with driving a car on the highway. When we are driving
on the highway, we want to keep a constant distance between us and the car in
front.  The constant distance can be considered the reference time. We may feel
that we are too close to or too far from the car in front. It can be considered
as that there is a time offset. Then we adjusting the speed of our car.  It is
actually indirectly adjust the distance between us and the car in front.  Note
that the time of the system clock increases at a certain frequency. The speed
of our car can be considered as the frequency of the system clock. The speed
difference between our car and the car in front can be considered as the
frequency offset.

\section{Definitions}%
\label{sec:clock_discipline_concepts}
To understand this chapter, we first introduce some definitions.  Note that it
seems that all definitions, architectures and algorithm names vary from one
part of an official document to the next.  It is as if the documents were
separated into two parts which were written by different people who did not
agree on their terminology.

\begin{enumerate}
    \item Phase\\
        In this chapter, \emph{phase} and time are used interchangeably.
    \item Frequency offset\\
        As described previously, \emph{frequency offset} is the difference
        between the frequency of the system clock and the frequency of the
        reference clock.
    \item Wander\\
        \emph{Wander} is the root-mean-square (RMS) of frequency offset
        differences.  It describes how stable the clock frequency is and is
        like jitter, which is the root-mean-square of phase offset difference.
    \item Variable frequency oscillator (VFO)\\
        The system clock, which has phase and frequency, is considered to be a
        VFO for the purposes of the NTP algorithms.
\end{enumerate}

\section{Overview}%
\label{sec:clock_discipline_overview}
Figure~\ref{fig:clock_discipline_arch}~\cite{redbook} shows the architecture of
the clock discipline process. We can see that there is a loop, which is the
same loop as in Figure~\ref{fig:architecture_overview}. The phase detector is
part of the peer process, which calculates offsets. The clock filter is the
combination of the clock filter algorithm in the peer process and algorithms in
the system process.  

Figure~\ref{fig:clock_discipline_arch} shows that the
clock discipline process is actually a feedback control system. The VFO
represents the system clock.  ``The variable $\theta_r$ represents the combined
server reference phase and $\theta_c$ represents the control phase of the
VFO''~\cite{redbook}. The signal $V_d$ represents the offset calculated in the
peer process after an update is received from the every server. Then the signal
$V_s$ is the system offset produced by the system process. 

After the signal
$V_s$ is passed to the phase/frequency prediction part, the predicted phase
adjustment $x$ and the predicted frequency $y$ are generated. Note that in the
book~\cite{redbook}, $x$ is first called ``phase adjustment'' and later
``phase''. Here the definitions are slightly modified based on the context of
the relevant paragraphs in~\cite{redbook}, although it makes no sense to refer
to one as an adjustment while not also referring to the other as an adjustment.
The clock adjust process ``runs at intervals of 1s in the NTP daemon or one
tick in the kernel''~\cite{redbook}, which generates the signal $V_c$. The VFO
frequency is controlled by $V_c$. Note here, ``one tick in the kernel'' means
one tick of a timer, which ticks once per second~\cite{rfc5905}.


% clock_discipline_arch.tex

\begin{figure}[htpb]
\begin{center}
\subfloat[clock discipline algorthm]{

\begin{tikzpicture}[scale=0.7, transform shape,
        squarednode/.style={rectangle, draw=black, very thick, minimum
        size=15mm, minimum width=30mm, align=center},
        circlenode/.style={circle, draw=black, very thick, minimum size=15mm},
        border/.style={-, draw=black, very thick, dashed},
    ]
    % Nodes
    \node[squarednode] (pd)                     
                    {Phase\\ detector};
    \node[squarednode] (cf)     [right=1cm of pd]   
                    {Clock filter};
    \node[squarednode] (pfp)    [below=2cm of cf]   
                    {Phase/freq\\ prediction};
    \node[squarednode] (ca)     [left=1cm of pfp]   
                    {Clock\\ adjust};
    \node[align=center](lf)     [above=of $(ca.east)!0.5!(pfp.west)$] 
                    {Loop filter};
    \node[circlenode]  (vfo)    [left=of $(pd.west)!0.5!(ca.west)$]   
                    {VFO};
    \node[align=center] (NTP)   [left=2cm of $(pd.north west)!0.5!(pd.west)$] 
                    {NTP};
    % lines
    \draw[-latex, thick] (pd) -- (cf) node[midway, above] {$V_d$};
    \draw[-latex, thick] (cf) -- ($(cf.east) + (1,0)$) 
                                    node[midway, above] {$V_s$} 
                               |- (pfp);
    \draw[-latex, thick] ($(pfp.west)!0.5!(pfp.north west)$) 
                      -- ($(ca.east)!0.5!(ca.north east)$) 
                                    node[midway, above] {$x$};
    \draw[-latex, thick] ($(pfp.west)!0.5!(pfp.south west)$) 
                      -- ($(ca.east)!0.5!(ca.south east)$) 
                                    node[midway, above] {$y$};

    \draw[-latex, thick] (ca) -| (vfo) node[pos=.3, above] {$V_c$};
    \draw[-latex, thick] (vfo) |- ($(pd.west)!0.5!(pd.south west)$) 
                        node[pos=.73, above] {$\theta_c-$};
    \draw[-latex, thick] (NTP) -- ($(pd.west)!0.5!(pd.north west)$) 
                        node[pos=.5, above] {$\theta_c+$};

    \draw[border] ($(ca.south west) + (-0.4, -0.4)$) 
               |- ($(pfp.north east) + (0.4, 1.2)$)
               |- ($(ca.south west) + (-0.4, -0.4)$);


\end{tikzpicture}
}

\subfloat[Phase/freq prediction] {
\begin{tikzpicture}[scale=0.7, transform shape,
        squarednode/.style={rectangle, draw=black, very thick, minimum
        size=15mm, minimum width=30mm, align=center},
        circlenode/.style={circle, draw=black, very thick, minimum size=15mm},
        border/.style={-, draw=black, very thick, dashed},
    ]
    % nodes
    \node[squarednode]  (pc)                    {Phase\\ correct};
    \node[squarednode]  (fp) [below=of pc]      {FLL\\ predict};
    \node[squarednode]  (pp) [below=of fp]      {PLL\\ predict};
    \node[align=center] (vs) [right=3cm of fp]  {$V_s$};
    \node[align=center] (x)  [left=of pc]       {$x$};
    \node[circlenode]  (sum) [left=of $(fp.west)!0.5!(pp.west)$] {$\Sigma$};
    \node[align=center] (y)  [left=of sum]      {$y$};

    % coordinates
    \coordinate (p1) at ( $(fp.east)!0.5!(vs.west)$ );

    % lines
    \draw[-latex, thick] (vs)--(p1)node{};
    \draw[-latex, thick] (p1)node{}|-(pc.east);
    \draw[-latex, thick] (p1)node{}--(fp.east);
    \draw[-latex, thick] (p1)node{}|-(pp.east);
    
    \draw[-latex, thick] (pc)--(x);
    \draw[-latex, thick] (fp)-|(sum) node[pos=.3, above] {$y_{FLL}$};
    \draw[-latex, thick] (pp)-|(sum) node[pos=.3, above] {$y_{PLL}$};

    \draw[-latex, thick] (sum)--(y);

\end{tikzpicture}
}
\end{center}
\caption{Clock discipline process}
\label{fig:clock_discipline_arch}
\end{figure}



% fig:clock_discipline_arch

\section{Phase/frequency prediction}%
\label{sec:phase_frequency_prediction}
There are two kinds of designs for phase/frequency prediction: phase-locked
loop (PLL) design and frequency-locked loop (FLL) design. In a PLL design, we
change the phase directly and the frequency is updated indirectly. In a FLL
design, we change the frequency directly and the phase is updated indirectly.
In the current version of NTP (version 4) the two types of designs are combined
together. 

Figure~\ref{fig:phase_frequency_prediction}~\cite{redbook} shows the process of
phase/frequency prediction. We have $x = V_s$ and $y = y + y_{PLL} + y_{FLL}$,
where
\begin{equation}
    y_{PLL} = \frac{V_s\mu}{\left(64T_c\right)^2},\
    y_{FLL} = \frac{V_s-x_r}{8\mu},
    \label{eq:y_pll}
\end{equation}
$\mu$ is the time since the last update and $x_r$ is the residual phase
error computed by the clock adjust process. This calculation is used to adjust
the clock frequency as will be seen in Section~\ref{sec:clock_state_machine}.

Otherwise, the frequency offset $\Phi$ is directly calculated by:
\begin{equation}
    \Phi = \frac{\Theta}{\mu}
    \label{eq:frequency_offset}
\end{equation}
This is used to step the frequency, as described
Section~\ref{sec:clock_state_machine}. 

%However, since the calculations are based on the result of the previous
%processes and calculations there are based on the assumption that the
%frequencies of client and servers are the same or close enough. This
%calculation and further changes about frequency will make it close to accurate
%one, but it needs time.


% clock_discipline_arch.tex

\begin{figure}[htpb]
\begin{center}
\begin{tikzpicture}[scale=0.7, transform shape,
        squarednode/.style={rectangle, draw=black, very thick, minimum
        size=15mm, minimum width=30mm, align=center},
        circlenode/.style={circle, draw=black, very thick, minimum size=15mm},
        border/.style={-, draw=black, very thick, dashed},
    ]
    % nodes
    \node[squarednode]  (pc)                    {Phase\\ correct};
    \node[squarednode]  (fp) [below=of pc]      {FLL\\ predict};
    \node[squarednode]  (pp) [below=of fp]      {PLL\\ predict};
    \node[align=center] (vs) [right=3cm of fp]  {$V_s$};
    \node[align=center] (x)  [left=of pc]       {$x$};
    \node[circlenode]  (sum) [left=of $(fp.west)!0.5!(pp.west)$] {$\Sigma$};
    \node[align=center] (y)  [left=of sum]      {$y$};

    % coordinates
    \coordinate (p1) at ( $(fp.east)!0.5!(vs.west)$ );

    % lines
    \draw[-latex, thick] (vs)--(p1)node{};
    \draw[-latex, thick] (p1)node{}|-(pc.east);
    \draw[-latex, thick] (p1)node{}--(fp.east);
    \draw[-latex, thick] (p1)node{}|-(pp.east);
    
    \draw[-latex, thick] (pc)--(x);
    \draw[-latex, thick] (fp)-|(sum) node[pos=.3, above] {$y_{FLL}$};
    \draw[-latex, thick] (pp)-|(sum) node[pos=.3, above] {$y_{PLL}$};

    \draw[-latex, thick] (sum)--(y);

\end{tikzpicture}
\end{center}
\caption{Phase/frequency prediction}
\label{fig:phase_frequency_prediction}
\end{figure}



% fig:phase_frequency_prediction

\section{Spikes and step control}%
\label{sec:spikes_and_step_control}
As we mentioned in Section~\ref{sub:clock_filter_algorithm}, the variance of
offset depends on the round-trip delay. We can observe that the performance of
NTP depends on the quality of the network connection. The peer/poll processes
are designed to maximize the performance of NTP while minimizing the impact on
network traffic, as well as minimizing errors due to delay. The system process
can find the most reliable source to which the system can synchronize. However,
there still may be some spikes due to network congestion. 

A \emph{noise gate} is used to deal with spikes. As mentioned at the beginning 
of this chapter, there are two thresholds, known as the \emph{step threshold}
and the
\emph{panic threshold}. If the offset exceeds the step threshold, but not the
panic threshold, NTP steps the time. With the noise gate, we make this decision
later since a spike may occur. The noise gate uses a ``watchdog counter'' and a
\emph{stepout threshold}. The counter counts the seconds since the first time
the offset exceeded the step threshold. If the offset becomes less than the step
threshold, the counter is set to zero. We step the time if the counter exceeds
the stepout threshold. Then the counter is set back to zero. The default value
of the stepout threshold is reported to be 900 seconds in~\cite{redbook}
and~\cite{rfc5905}, but is 300 seconds in~\cite{source_code}
and~\cite{clock_state_machine}. 

With the noise gate, we can avoid stepping time when a spike occurs. If the
time offset is large, there should be continuous ``spikes''. Consequently the
time is stepped when the spikes
occur continuously longer than the stepout threshold. We can see
that NTP avoids stepping the time unless there is sufficient evidence that the
time offset must be stepped. 

\section{Intrinsic frequency offset}%
\label{sec:intrinsic_frequency_offset}
As mentioned in Section~\ref{sec:adjusting_system_time}, each system clock has an
intrinsic frequency which may not be accurate. The accurate frequency should be
1s/s, which means that the clock should go forward one second every second. The
intrinsic frequency may be some other number. In that case, there is a
frequency offset. Note that in the implementation of NTP\null, the word
``frequency'' is used to mean ``frequency offset''~\cite{source_code}. 

Suppose that at some time the NTP daemon stepped the system time After a period
of time, there may be another time offset because of the frequency offset. NTP
can determine the frequency offset from the current offset and the time period
length.  However, this calculation may take many hours of time~\cite{redbook},
which is called the \emph{training time}. 

One solution to reduce the training time is to record the frequency offset into a file,
which is called \emph{frequency file} or \emph{drift file}. On Ubuntu 18.04.2
LTS, with the NTP
daemon program \verb|ntpd| version 4.2.8p10 and the default \verb|ntpd|
configuration (which is found in \verb|/etc/ntp.conf|), the file is
\verb|/var/lib/ntp/ntp.drift|. The path of the frequency file can be set by
using \verb|driftfile| command in the configuration file
(\verb|/etc/ntp.conf|). The \verb|ls -l| command shows that the owner of
the file is \verb|ntp| in group \verb|ntp| and the permission of the file is
644. When the NTP daemon is running, the file is updated hourly.

This solution can reduce the training time after the system is rebooted and
connected to a congested network. After rebooting, the frequency can be
initialized based on the result of previous executions of the daemon.  

If we start up the NTP daemon without the frequency file, it first changes the
time (step or adjust, based on the offset) and then starts the watchdog counter
mentioned in Section~\ref{sec:spikes_and_step_control} after a valid update
arrives. Then the time will not be changed until the counter exceeds the
stepout threshold, at which time, the frequency is stepped and then the time can
be changed.

\section{Clock state machine}%
\label{sec:clock_state_machine}
The NTP daemon uses a clock state machine which combines the ideas mentioned in
the previous three sections.
Table~\ref{tab:clock_state_machine_transition_function}, which is based on the
same table in~\cite{redbook} with some changes, shows the states and transition
functions. Note that in the same table in~\cite{rfc5905}, there is an apparent
error in that the cell where the state is \verb|FREQ| and $|\Theta| >$
\verb|STEP|, it should be ``step time'' instead of ``adjust time''.

In the table, $\Theta$ is the system offset, \verb|STEP| and \verb|STEPOUT| are
the step threshold and stepout threshold, the notation \verb|>| indicates the
next state and $\mu$ is the watchdog counter mentioned in
Section~\ref{sec:spikes_and_step_control} and
Section~\ref{sec:intrinsic_frequency_offset}. When the NTP daemon starts up,
the machine enters either the \verb|FSET| state or the \verb|NSET| state
depending on whether or not there is a frequency file. Note that in all cells
which contain an if statement, the frequency and time are only changed inside
of the else branch.

\begin{table}[ht]
    \centering
    \caption{Clock State Machine Transition Function}
    \label{tab:clock_state_machine_transition_function}
    \begin{tabular}{|c|p{43mm}|p{43mm}|c|}
        \hline
        State &\hfil $|\Theta|\le$ \verb|STEP| & \hfil $|\Theta|>$ \verb|STEP| 
        & Comments \\
        \hline
        \verb|NSET| & \verb|>FREQ|; adjust time & \verb|>FREQ|; step time 
            & No frequency file \\
        \hline
        \verb|FSET| & \verb|>SYNC|; adjust time & \verb|>SYNC|; step time
            & Frequency file \\
        \hline
        \verb|SPIK| & \verb|>SYNC|; adjust freq; adjust time 
        & if ($\mu <$ \verb|STEPOUT|) \verb|>SPIK| \newline 
        else \verb|>SYNC|; step freq; step time;
        & Outlier detected \\
        \hline
        \verb|FREQ| & if ($\mu <$ \verb|STEPOUT|) \verb|>FREQ| \newline 
        else \verb|>SYNC|; step freq; adjust time;
        & if ($\mu <$ \verb|STEPOUT|) \verb|>FREQ| \newline 
        else \verb|>SYNC|; step freq; step time;
        & Initial frequency \\
        \hline
        \verb|SYNC| & \verb|>SYNC|; adjust freq; adjust time;
        & if ($\mu <$ \verb|STEPOUT|) \verb|>SPIK| \newline 
        else \verb|>SYNC|; step freq; step time; & Normal operation \\
        \hline
    \end{tabular}
\end{table}

\section{Clock adjust process}%
\label{sec:clock_adjust_process}
The functionality of the clock adjust process depends on how the time is
changed.  If the time is stepped, a system call like \verb|settimeofday()| is
called. If the time is slewed, the \verb|adjtime()| or \verb|ntp_adjtime()|
system call is used every second and a phase increment $z=\frac{x}{16T_c}$
is passed to the system call, where $x$ is computed as in
Section~\ref{sec:phase_frequency_prediction}, and $T_c$ is the length of the
poll interval.  Then $x$ is updated as $x=x-z$, where the updated $x$ is
called the residual phase error.

\section{Poll interval control}%
\label{sec:poll_interval_control}
As mentioned in Chapter~\ref{cha:peer/poll_processes}, poll intervals are
maintained by the clock discipline process. Note that the algorithm described
below is a combination of descriptions from various NTP documents; the
descriptions from~\cite{redbook}, \cite{poll_process} and~\cite{rfc5905} are
all different from each other. Here the description is based on the
implementation~\cite{source_code}.

Recall that the length of the poll interval is called the time constant,
represented by $T_c$, and we have $T_c = 2^\uptau$ where $\uptau$ is called the
poll exponent. The basic idea of the maintenance is: when the NTP daemon does
not need to change the system clock a ``large'' amount, the poll interval
should be increased; otherwise it should be decreased.

When the daemon starts up, the poll interval is not adjusted in the first
\verb|STEPOUT| seconds, where \verb|STEPOUT| is the stepout threshold.
To maintain the poll interval, we need a statistic: \emph{clock jitter} $\psi$.
It is initialized to the precision of the system clock when the daemon starts
up and will not change in the first \verb|STEPOUT| seconds. After that, when
an update of the system offset $\Theta$ occurs, the jitter is updated. If
$\Theta > $ \verb|STEP|, the clock jitter is reset to its initial value;
otherwise the clock jitter is updated as follows:
\begin{equation}
    \psi = \sqrt{ \frac{7}{8}\psi_{old}^2 + \frac{1}{8}\left(\Theta -
    \Theta_{old}\right)^2 }
    \label{eq:clock_jitter}
\end{equation}
where $\psi_{old}$ is the old clock jitter and $\Theta_{old}$ is the last
updated system offset. 

NTP uses a counter called the \emph{jiggle counter}, which is initialized to
zero. When an update of the system offset occurs, if the system offset exceeds
the step interval, it is also set to zero. If the system offset does not exceed
the step interval, NTP operates as follows:
if $\Theta < 4 \times \psi$, the counter increases by $\uptau$; 
otherwise the counter decreases by $2\uptau$.
When the jiggle counter is larger than 30, we increase $\uptau$ by 1; when it
is less than $-30$, we decrease $\uptau$ by 1. After we change $\uptau$, the
jiggle counter will be set to zero. ``In effect, the algorithm has a relatively
slow reaction to good news, but a relatively fast reaction to bad
news''~\cite{poll_process}.



