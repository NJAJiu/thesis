
% clock_discipline.tex

\chapter{Clock discipline process}%
\label{cha:clock_discipline_process}
Now we are going to adjust the system clock of the client. As mentioned in 
Section~\ref{sec:adjusting_system_time}, we want the system time to be changed
``smoothly'', which means that we want to change it a little bit every time and 
repeat it many times over a acceptable period of time. There are two
thresholds: \emph{step} and \emph{panic}. When the offset exceeds panic
threshold, which is 1000s by default, we do not adjust the time. If the
offset is less than the panic threshold but more than the step threshold,
which is 128 ms by default, we change the time in the ``non-smooth'' way. We
say that we the time is stepped. If the offset is less than the step
threshold, we change the time ``smoothly'', we say that the time is slewed or
adjusted.

There is an example described in \cite{redbook}, we can compare adjusting
system clock with driving a car on the highway. When we are driving on the
highway, we want to keep a constant distance between us and the car in front.
The constant distance can be considered as the reference time. We may feel that
we are too close to or too far from the car in front. It can be considered as
the time offset. Then we adjust the speed of our car. It is actually indirectly
adjust the distance between us and the car in front. Note that the time of
the system clock increases at a certain frequency. The speed of our car can be
considered as the frequency of the system clock. The speed difference between
our car and the car in front can be considered as the frequency offset.

To understand this chapter, we introduce some concepts first. Note that it
seems that all concepts, architectures and algorithm names are changed here in
all official documents. It is like that documents are separated into two parts
which are written by different people.

\section{Concepts}%
\label{sec:clock_discipline_concepts}
\begin{enumerate}
    \item Phase\\
        In this chapter, phase and time are used interchangeably.
    \item Frequency offset\\
        As described previously, frequency offset is the difference between the
        frequency of the system clock and the frequency of the reference clock.
    \item Wander\\
        Wander is the root-mean-square (RMS) of frequency offset difference. It
        describes how stable the clock frequency is and is like jitter, which
        is the root-mean-square of phase offset difference.
    \item Variable frequency oscillator(VFO)\\
        It can be considered as the system clock, which has phase and
        frequency.
\end{enumerate}

\section{Overview}%
\label{sec:clock_discipline_overview}
Figure~\ref{fig:clock_discipline_arch} shows the architecture of clock
discipline process. We can see that there is a loop, which is the same loop in 
Figure~\ref{fig:architecture_overview}. The phase detector is part of peer
process, which calculates offsets. The clock filter is the combination of
the clock filter algorithm in peer process and algorithms in system process.
Figure~\ref{fig:clock_discipline_arch} shows that the clock discipline process
is actually a feedback control system. The VFO represents the system clock.
``The variable $\theta_r$ represents the combined server reference phase and
$\theta_c$ represents the control phase of the VFO.''~\cite{redbook} The signal
$V_d$ represents the offset calculated in peer process after a update received
from the sever. Then the signal $V_s$ is the system offset produced by the
system process. After the signal $V_s$ is passed to the phase/frequency
prediction part, the predicted phase adjustment $x$ and the predicted frequency
$y$ are generated. Note that in the book~\cite{redbook}, $x$ is first called
``phase adjustment'' and later ``phase''. Here the definitions are slightly
modified based on the context of relative paragraphs in~\cite{redbook},
although it makes no sense to call one adjustment while do not call the other
one. The clock adjust process ``runs at intervals of 1s in the NTP daemon or
one tick in the kernel''~\cite{redbook}, which generates the signal $V_c$. The
VFO frequency $\omega_c$ is controlled by $V_c$.


% clock_discipline_arch.tex

\begin{figure}[htpb]
\begin{center}
\begin{tikzpicture}[scale=0.7, transform shape,
        squarednode/.style={rectangle, draw=black, very thick, minimum
        size=15mm, minimum width=30mm, align=center},
        circlenode/.style={circle, draw=black, very thick, minimum size=15mm},
        border/.style={-, draw=black, very thick, dashed},
    ]
    % Nodes
    \node[squarednode] (pd)                     
                    {Phase\\ detector};
    \node[squarednode] (cf)     [right=1cm of pd]   
                    {Clock filter};
    \node[squarednode] (pfp)    [below=2cm of cf]   
                    {Phase/freq\\ prediction};
    \node[squarednode] (ca)     [left=1cm of pfp]   
                    {Clock\\ adjust};
    \node[align=center](lf)     [above=of $(ca.east)!0.5!(pfp.west)$] 
                    {Loop filter};
    \node[circlenode]  (vfo)    [left=of $(pd.west)!0.5!(ca.west)$]   
                    {VFO};
    \node[align=center] (NTP)   [left=2cm of $(pd.north west)!0.5!(pd.west)$] 
                    {NTP};
    % lines
    \draw[-latex, thick] (pd) -- (cf) node[midway, above] {$V_d$};
    \draw[-latex, thick] (cf) -- ($(cf.east) + (1,0)$) 
                                    node[midway, above] {$V_s$} 
                               |- (pfp);
    \draw[-latex, thick] ($(pfp.west)!0.5!(pfp.north west)$) 
                      -- ($(ca.east)!0.5!(ca.north east)$) 
                                    node[midway, above] {$x$};
    \draw[-latex, thick] ($(pfp.west)!0.5!(pfp.south west)$) 
                      -- ($(ca.east)!0.5!(ca.south east)$) 
                                    node[midway, above] {$y$};

    \draw[-latex, thick] (ca) -| (vfo) node[pos=.3, above] {$V_c$};
    \draw[-latex, thick] (vfo) |- ($(pd.west)!0.5!(pd.south west)$) 
                        node[pos=.73, above] {$\theta_c-$};
    \draw[-latex, thick] (NTP) -- ($(pd.west)!0.5!(pd.north west)$) 
                        node[pos=.5, above] {$\theta_r+$};

    \draw[border] ($(ca.south west) + (-0.4, -0.4)$) 
               |- ($(pfp.north east) + (0.4, 1.2)$)
               |- ($(ca.south west) + (-0.4, -0.4)$);


\end{tikzpicture}
\end{center}
\caption{Clock discipline process}
\label{fig:clock_discipline_arch}
\end{figure}



% fig:clock_discipline_arch


\section{Phase/frequency prediction}%
\label{sec:phase_frequency_prediction}
There are two kinds of designs: phase-locked loop (PLL) design and
frequency-locked loop (FLL) design. In a PLL design, we change the phase
directly and the frequency is updated indirectly. In a FLL design, we change
the frequency directly and the phase is updated indirectly. In the current version
of NTP (version 4) the two types of designs are combined together. 

Figure~\ref{fig:phase_frequency_prediction} shows the process of phase/frequency
prediction. We have $x = V_s$ and $y = y + y_{PLL} + y_{FLL}$, where
\begin{equation}
    y_{PLL} = \frac{V_s\mu}{\left(64T_c\right)^2},
    y_{FLL} = \frac{V_s-x_r}{8\mu}
    \label{eq:y_pll}
\end{equation}
note that $\mu$ is the time since last update and $x_r$ is the residual phase
error computed by the clock adjust process. This calculation is used to adjust
frequency and we will see that in Section~\ref{sec:clock_state_machine}.

Otherwise, the frequency offset $\Phi$ is directly calculated by:
\begin{equation}
    \Phi = \frac{\Theta}{\mu}
    \label{eq:frequency_offset}
\end{equation}
This is used to step frequency in Section~\ref{sec:clock_state_machine}. 

However, since the calculations are based on the result of previous processes
and calculations there are based on the assumption that the frequencies of client
and servers are the same or close enough. This calculation and further changes
about frequency will make it close to accurate one, but it need time.


% clock_discipline_arch.tex

\begin{figure}[htpb]
\begin{center}
\begin{tikzpicture}[scale=1.0, transform shape,
        squarednode/.style={rectangle, draw=black, very thick, minimum
        size=15mm, minimum width=30mm, align=center},
        circlenode/.style={circle, draw=black, very thick, minimum size=15mm},
        border/.style={-, draw=black, very thick, dashed},
    ]
    % nodes
    \node[squarednode]  (pc)                    {Phase\\ correct};
    \node[squarednode]  (fp) [below=of pc]      {FLL\\ predict};
    \node[squarednode]  (pp) [below=of fp]      {PLL\\ predict};
    \node[align=center] (vs) [right=3cm of fp]  {$V_s$};
    \node[align=center] (x)  [left=of pc]       {$x$};
    \node[circlenode]  (sum) [left=of $(fp.west)!0.5!(pp.west)$] {$\Sigma$};
    \node[align=center] (y)  [left=of sum]      {$y$};

    % coordinates
    \coordinate (p1) at ( $(fp.east)!0.5!(vs.west)$ );

    % lines
    \draw[-latex, thick] (vs)--(p1)node{};
    \draw[-latex, thick] (p1)node{}|-(pc.east);
    \draw[-latex, thick] (p1)node{}--(fp.east);
    \draw[-latex, thick] (p1)node{}|-(pp.east);
    
    \draw[-latex, thick] (pc)--(x);
    \draw[-latex, thick] (fp)-|(sum) node[pos=.3, above] {$y_{FLL}$};
    \draw[-latex, thick] (pp)-|(sum) node[pos=.3, above] {$y_{PLL}$};

    \draw[-latex, thick] (sum)--(y);

\end{tikzpicture}
\end{center}
\caption{Phase/frequency prediction}
\label{fig:phase_frequency_prediction}
\end{figure}



% fig:phase_frequency_prediction

\section{Spikes and step control}%
\label{sec:spikes_and_step_control}
As we mentioned in Section~\ref{sub:clock_filter_algorithm}, the variance of
offset depends on the round trip delay. We should say that the performance of
NTP depends on the quality of network connection. The peer/poll processes are
designed to maximize the performance of NTP while minimize the impact on
network traffic and to minimize errors due to delay. The system process can
produce the most reliable source that the client can synchronize to. However, 
there still may be some spikes due to network congestion. 

A \emph{noise gate} is used to deal with spikes. As mentioned at the beginning 
of this chapter, there are two thresholds, which are \emph{step threshold} and 
\emph{panic threshold}. If the offset exceeds the step threshold, but not the
panic threshold, we should step the time. With noise gate, we make the decision
later since it may be a spike. Noise gate uses a watchdog counter and a
\emph{stepout threshold}. The counter counts the seconds since the first time
the offset exceeds the step threshold. If the offset becomes less than the step
threshold, the counter is set to zero. We step the time if the counter exceeds
the stepout threshold. The default value of stopout threshold is 900s
in~\cite{redbook} and~\cite{rfc5905}, but is 300s in~\cite{source_code}
and~\cite{clock_state_machine}. Then the counter is set back to zero.

With the noise gate, we can avoid stepping time when a spike occurs. If the
time offset is large, there should be continuous ``spikes''. So we step the
time when them occurs longer than the stepout threshold. We can see that NTP
does not want to step the time unless there is sufficient evidence.

\section{Intrinsic frequency offset}%
\label{sec:intrinsic_frequency_offset}
As mentioned in~\ref{sec:adjusting_system_time}, each system clock has an
intrinsic frequency, it may not be accurate. The accurate frequency should be
1s/s, which means that the clock should go forward one second every second. The
intrinsic frequency may be some other number. In that case, there is a
frequency offset. Note that in the implementation of NTP\null, they use
frequency to represent frequency offset. 

Imagine that we stepped the system time at some time. After a period of time,
there may be another time offset because of the frequency offset. NTP can
determine the frequency offset from current offset and the time period length.
However, it may take hours of time~\cite{redbook}. 

One solution of this problem is to record the frequency offset into a file,
which is called \emph{frequency file}. On Ubuntu 18.04.2 LTS, with the NTP
daemon program \verb|ntpd| version 4.2.8p10, the file is
\verb|/var/lib/ntp/ntp.drift|. The \verb|ls -l| command shows that the owner of
the file is \verb|ntp| in group \verb|ntp| and the permission of the file is
644. When the ntp daemon is running, the file is updated in hourly interval.

This solution can reduce the training time after the system is reboot and
connected into a congest network. After rebooting, the frequency can be
initialize based on the result of previously execution of the daemon.  

If we start up the NTP daemon without the frequency file, it first changes the
time (step or adjust, based on the offset) and then starts the watchdog counter
mentioned in Section~\ref{sec:spikes_and_step_control} after a valid update
arrives. Then the time will not be changed until the counter exceeds the
stepout threshold.  After that, the frequency is stepped and then the time can
be changed.

\section{Clock state machine}%
\label{sec:clock_state_machine}
There is a clock state machine which combines the ideas mentioned before. 
The table\ref{tab:clock_state_machien_transition_function} shows the states
and transitions, which is based on the same table in~\cite{redbook} with some
changes. Note that in the same table in~\cite{rfc5905}, there is an apparent
error that in the cell where the state is \verb|FREQ| and $|\Theta| >$
\verb|STEP|, it should be ``step time'' instead of ``adjust time''.

\begin{table}[ht]
    \centering
    \caption{Clock State Machine Transition Function}
    \label{tab:clock_state_machien_transition_function}
    \begin{tabular}{|c|p{43mm}|p{43mm}|c|}
        \hline
        State &\hfil $|\Theta|\le$ \verb|STEP| & \hfil $|\Theta|>$ \verb|STEP| 
        & Comments \\
        \hline
        \verb|NSET| & \verb|>FREQ|; adjust time & \verb|>FREQ|; step time 
            & No frequency file \\
        \hline
        \verb|FREQ| & \verb|>SYNC|; adjust time & \verb|>SYNC|; step time
            & Frequency file \\
        \hline
        \verb|SPIK| & \verb|>SYNC|; adjust freq; adjust time 
        & if ($\mu <$ \verb|STEPOUT|) \verb|>SPIK| \newline 
        else \verb|>SYNC|; step freq; step time;
        & Outlier detected \\
        \hline
        \verb|FREQ| & if ($\mu <$ \verb|STEPOUT|) \verb|>FREQ| \newline 
        else \verb|>SYNC|; step freq; adjust time;
        & if ($\mu <$ \verb|STEPOUT|) \verb|>FREQ| \newline 
        else \verb|>SYNC|; step freq; step time;
        & Initial frequency \\
        \hline
        \verb|SYNC| & \verb|>SYNC|; adjust freq; adjust time;
        & if ($\mu <$ \verb|STEPOUT|) \verb|>SPIK| \newline 
        else \verb|>SYNC|; step freq; step time; & Normal operation \\
        \hline
    \end{tabular}
\end{table}

In the table, $\Theta$ is the system offset, \verb|STEP| and \verb|STEPOUT| are
the step threshold and stepout threshold, the notation \verb|>| indicates the
next state, $\mu$ is the watchdog counter mentioned in
Section~\ref{sec:spikes_and_step_control} and
Section~\ref{sec:intrinsic_frequency_offset}. After the NTP daemon starts up,
the machine enters either \verb|NSET| state or \verb|FSET| state depends on
whether there is a frequency file. Note that in all cells which contain if
statement, the frequency and time are only changed inside of the else branch.

\section{Clock adjust process}%
\label{sec:clock_adjust_process}
The functionality of clock adjust process depends on how the time is changed.
If the time is stepped, a system call like \verb|settimeofday()| is called. If
the time is slewed, the \verb|adjtime()| or \verb|ntp_adjtime()| system call is
used every one second and a phase increment $z=\frac{x}{16T_c}$ is passed to
the system call, where $x$ is computed in
Section~\ref{sec:phase_frequency_prediction}. Then the $x$ is updated as
$x=x-z$ where the updated $x$ is called the residual phase error.

\section{Poll interval control}%
\label{sec:poll_interval_control}
As mentioned in Chapter~\ref{cha:peer/poll_processes}, poll intervals are
maintained by the clock discipline process. Note that the algorithm described
below is a combination of descriptions from various NTP documents, the
descriptions from~\cite{redbook}, \cite{poll_process} and~\cite{rfc5905} are
ALL different from each other. Here the description is based on the
implementation~\cite{source_code}.

Recall that the length of poll interval is called time constant, represented by
$T_c$ and we have $T_c = 2^\uptau$ where $\uptau$ is called time exponent.

When the daemon starts up, we do nothing about poll intervals in the first
\verb|STEPOUT| seconds, where \verb|STEPOUT| is the stepout interval.
To maintain the poll interval, we need a statistic: \emph{clock jitter} $\psi$.
It is initialize to the precision of system clock when the daemon starts up and
will not change after \verb|STEPOUT| seconds. After that, when a update of
system offset $\Theta$ comes, the jitter is updated. If $\Theta > $
\verb|STEP|, the clock jitter is changed into its initial value; otherwise the
clock jitter is updated as:
\begin{equation}
    \psi = \sqrt{ \frac{7}{8}\psi_{old}^2 + \frac{1}{8}\left(\Theta -
    \Theta_{old}\right)^2 }
    \label{eq:clock_jitter}
\end{equation}
where $\psi_{old}$ is the old clock jitter and $\Theta_{old}$ is the last
updated system offset. 

We need a counter called \emph{jiggle counter}, which is initialize to zero. If
the system offset exceeds the step interval, it is also set to zero. When the
system offset does not exceed the step interval:
if $\Theta < 4 \times \psi$, the counter increases by $\uptau$; 
otherwise the counter decreases by $2\uptau$.
When the jiggle counter is larger than 30, we increase $\uptau$ by 1; when it is
less than $-30$, we decrease $\uptau$ by 1. After we change $\uptau$, the
jiggle counter will be set to zero. ``In effect, the algorithm has a relatively
slow reaction to good news, but a relatively fast reaction to bad
news.''~\cite{poll_process}



