
% clock_discipline.tex

\chapter{Clock discipline process}%
\label{cha:clock_discipline_process}
Now we are going to adjust the system clock of the client. As mentioned in 
Section~\ref{sec:adjusting_system_time}, we want the system time to be changed
``smoothly'', which means that we want to change it a little bit every time and 
repeat it many times over a acceptable period of time. There are two
thresholds: \emph{step} and \emph{panic}. When the offset exceeds panic
threshold, which is 1000s by default, we do not adjust the time. If the
offset is less than the panic threshold but more than the step threshold,
which is 128 ms by default, we change the time in the ``non-smooth'' way. We
say that we the time is stepped. If the offset is less than the step
threshold, we change the time ``smoothly'', we say that the time is slewed or
adjusted.

There is an example described in \cite{redbook}, we can compare adjusting
system clock with driving a car on the highway. When we are driving on the
highway, we want to keep a constant distance between us and the car in front.
The constant distance can be considered as the reference time. We may feel that
we are too close to or too far from the car in front. It can be considered as
the time offset. Then we adjust the speed of our car. It is actually indirectly
adjust the distance between us and the car in front. Note that the time of
the system clock increases at a certain frequency. The speed of our car can be
considered as the frequency of the system clock. The speed difference between
our car and the car in front can be considered as the frequency offset.

To understand this chapter, we introduce some concepts first. Note that it
seems that all concepts, architectures and algorithm names are changed here in
all official documents. It is like that documents are separated into two parts
which are written by different people.

\section{Concepts}%
\label{sec:clock_discipline_concepts}
\begin{enumerate}
    \item Phase\\
        In this chapter, phase and time are used interchangeably.
    \item Frequency offset\\
        As described previously, frequency offset is the difference between the
        frequency of the system clock and the frequency of the reference clock.
    \item Wander\\
        Wander is the root-mean-square (RMS) of frequency offset difference. It
        describes how stable the clock frequency is and is like jitter, which
        is the root-mean-square of phase offset difference.
    \item Variable frequency oscillator(VFO)\\
        It can be considered as the system clock, which has phase and
        frequency.
\end{enumerate}

\section{Overview}%
\label{sec:clock_discipline_overview}
Figure~\ref{fig:clock_discipline_arch} shows the architecture of clock
discipline process. We can see that there is a loop, which is the same loop in 
Figure~\ref{fig:architecture_overview}. The phase detector is part of peer
process, which calculates offsets. The clock filter is the combination of
the clock filter algorithm in peer process and algorithms in system process.
Figure~\ref{fig:clock_discipline_arch} shows that the clock discipline process
is actually a feedback control system. The VFO represents the system clock.
``The variable $\theta_r$ represents the combined server reference phase and
$\theta_c$ represents the control phase of the VFO.''~\cite{redbook} The signal
$V_d$ represents the offset calculated in peer process after a update received
from the sever. Then the signal $V_s$ is the system offset produced by the
system process. After the signal $V_s$ is passed to the phase/frequency
prediction part, the predicted phase adjustment $x$ and the predicted frequency
$y$ are generated. Note that in the book~\cite{redbook}, $x$ is first called
``phase adjustment'' and later ``phase''. Here the definitions are slightly
modified based on the context of relative paragraphs in~\cite{redbook},
although it makes no sense to call one adjustment while do not call the other
one. The clock adjust process ``runs at intervals of 1s in the NTP deamon or
one tick in the kernel''~\cite{redbook}, which generates the signal $V_c$. The
VFO frequency $\omega_c$ is controlled by $V_c$.


% clock_discipline_arch.tex

\begin{figure}[htpb]
\begin{center}
\begin{tikzpicture}[scale=0.7, transform shape,
        squarednode/.style={rectangle, draw=black, very thick, minimum
        size=15mm, minimum width=30mm, align=center},
        circlenode/.style={circle, draw=black, very thick, minimum size=15mm},
        border/.style={-, draw=black, very thick, dashed},
    ]
    % Nodes
    \node[squarednode] (pd)                     
                    {Phase\\ detector};
    \node[squarednode] (cf)     [right=1cm of pd]   
                    {Clock filter};
    \node[squarednode] (pfp)    [below=2cm of cf]   
                    {Phase/freq\\ prediction};
    \node[squarednode] (ca)     [left=1cm of pfp]   
                    {Clock\\ adjust};
    \node[align=center](lf)     [above=of $(ca.east)!0.5!(pfp.west)$] 
                    {Loop filter};
    \node[circlenode]  (vfo)    [left=of $(pd.west)!0.5!(ca.west)$]   
                    {VFO};
    \node[align=center] (NTP)   [left=2cm of $(pd.north west)!0.5!(pd.west)$] 
                    {NTP};
    % lines
    \draw[-latex, thick] (pd) -- (cf) node[midway, above] {$V_d$};
    \draw[-latex, thick] (cf) -- ($(cf.east) + (1,0)$) 
                                    node[midway, above] {$V_s$} 
                               |- (pfp);
    \draw[-latex, thick] ($(pfp.west)!0.5!(pfp.north west)$) 
                      -- ($(ca.east)!0.5!(ca.north east)$) 
                                    node[midway, above] {$x$};
    \draw[-latex, thick] ($(pfp.west)!0.5!(pfp.south west)$) 
                      -- ($(ca.east)!0.5!(ca.south east)$) 
                                    node[midway, above] {$y$};

    \draw[-latex, thick] (ca) -| (vfo) node[pos=.3, above] {$V_c$};
    \draw[-latex, thick] (vfo) |- ($(pd.west)!0.5!(pd.south west)$) 
                        node[pos=.73, above] {$\theta_c-$};
    \draw[-latex, thick] (NTP) -- ($(pd.west)!0.5!(pd.north west)$) 
                        node[pos=.5, above] {$\theta_r+$};

    \draw[border] ($(ca.south west) + (-0.4, -0.4)$) 
               |- ($(pfp.north east) + (0.4, 1.2)$)
               |- ($(ca.south west) + (-0.4, -0.4)$);


\end{tikzpicture}
\end{center}
\caption{Clock discipline process}
\label{fig:clock_discipline_arch}
\end{figure}



% fig:clock_discipline_arch

\section{PLL and FLL}%
\label{sec:pll_and_fll}
There are two kinds of designs: phase-locked loop (PLL) design and
frequency-locked loop (FLL) design. In a PLL design, we change the phase
directly and the frequency is updated indirectly. In a FLL design, we change
the frequency directly and the phase is updated indirectly.

\section{Phase/frequency prediction}%
\label{sec:phase_frequency_prediction}
Figure~\ref{fig:phase_frequency_prediction} shows the detail of phase/frequency
prediction.


% clock_discipline_arch.tex

\begin{figure}[htpb]
\begin{center}
\begin{tikzpicture}[scale=1.0, transform shape,
        squarednode/.style={rectangle, draw=black, very thick, minimum
        size=15mm, minimum width=30mm, align=center},
        circlenode/.style={circle, draw=black, very thick, minimum size=15mm},
        border/.style={-, draw=black, very thick, dashed},
    ]
    % nodes
    \node[squarednode]  (pc)                    {Phase\\ correct};
    \node[squarednode]  (fp) [below=of pc]      {FLL\\ predict};
    \node[squarednode]  (pp) [below=of fp]      {PLL\\ predict};
    \node[align=center] (vs) [right=3cm of fp]  {$V_s$};
    \node[align=center] (x)  [left=of pc]       {$x$};
    \node[circlenode]  (sum) [left=of $(fp.west)!0.5!(pp.west)$] {$\Sigma$};
    \node[align=center] (y)  [left=of sum]      {$y$};

    % coordinates
    \coordinate (p1) at ( $(fp.east)!0.5!(vs.west)$ );

    % lines
    \draw[-latex, thick] (vs)--(p1)node{};
    \draw[-latex, thick] (p1)node{}|-(pc.east);
    \draw[-latex, thick] (p1)node{}--(fp.east);
    \draw[-latex, thick] (p1)node{}|-(pp.east);
    
    \draw[-latex, thick] (pc)--(x);
    \draw[-latex, thick] (fp)-|(sum) node[pos=.3, above] {$y_{FLL}$};
    \draw[-latex, thick] (pp)-|(sum) node[pos=.3, above] {$y_{PLL}$};

    \draw[-latex, thick] (sum)--(y);

\end{tikzpicture}
\end{center}
\caption{Phase/frequency prediction}
\label{fig:phase_frequency_prediction}
\end{figure}



% fig:phase_frequency_prediction

\section{Spikes and step control}%
\label{sec:spikes_and_step_control}
As we mentioned in Section~\ref{sub:clock_filter_algorithm}, the variance of
offset depends on the round trip delay. We should say that the performance of
NTP depends on the quality of network connection. The peer/poll processes are
designed to maximize the performance of NTP while minimize the impact on
network traffic and to minimize errors due to delay. The system process can
produce the most reliable source that the client can synchronize to. However, 
there still may be some spikes due to network congestion. 

A \emph{noise gate} is used to deal with spikes. As mentioned at the beginning 
of this chapter, there are two thresholds, which are \emph{step threshold} and 
\emph{panic threshold}. If the offset exceeds step threshold, but not the panic
threshold, we should step the time. With noise gate, we make the decision later
since it may be a spike. Noise gate uses a watchdog counter and a
\emph{stepout threshold}. The counter counts the seconds since the first time 
the offset exceeds the step threshold. If the offset becomes less than the
step threshold, the counter is set to zero. We step the time if the counter
exceeds the stepout threshold. The default value of stopout threshold is 900s
in~\cite{redbook} and~\cite{rfc5905}, but is 300s in (note: cite here,
implementation and the clock state machine webpage). Then the counter is set back
to zero.

\section{Clock state machine}%
\label{sec:clock_state_machine}


\section{Poll interval control}%
\label{sec:poll_interval_control}


