% introduction.tex

% time used in daily life
% computer, even embed system keeps track of time
% different requirement of time accuracy
% why it is hard to keep time
% Note: definition of UTC
% Dr. Zhang's advise: add contribution

% why it is important
% why it is hard
% the increasing requirement of accurate time 
% computer clock structure

\chapter{Introduction}
Time is frequently used in our daily life. For example, consider a young
woman who works in a company. On weekdays, she may wake up at 7 am, get to her
office before 9 am, have a lunch break from 11:30 to 1:30, go back home at 5
pm, have dinner at 7 pm, and finally go to sleep at 11 pm. Some of these times
are based on her personal habit, while some are required by the company she
works for.  Some of them are fixed, for example, she must get to the office
before 9 am, otherwise her salary might be docked. Some of them may differ
everyday, such as the time she has dinner. Sometimes people get dinner not
because they are hungry, but it is the time to do it. The schedule may be
different among people, but the requirement of knowing the time is the same.

There are various ways to let people know the time: clocks and watches are
commonly used. In old days, people used huge clocks to let everyone who can see
such a clock keep track of time by frequently looking at it. Even today, wrist
watches are still popular, since they make it easy for people to know the
current time, where the only requirement is to reset the time periodically.

In modern life, computers are widely used. Most computers, including
smartphones, have internal clocks to keep track of time. For computer programs, 
time is used for many different purposes. A program which makes log files needs
to know the time when an event to be recorded happened. Some programs can
change the background color of the screen based upon the current time. For
example, there are various programs which reduce the amount of blue color
emitted by the screen during night time to protect users' eyes.  Smartphones
can be considered as small computers and they have become one of the main
sources that people get time from. Even simple digital systems like parking
meters or home thermostats have internal clocks. A parking meter needs a clock
to determine when the parking expires. People may want their heaters turned
down when they are not at home or sleeping and turned up before they get home
and wake up. A home thermostat needs a clock so that it can automatically do
these things.

In practice, different accuracies are required for different purposes. 
In some cases, having the system clock of a computer accurate to one
minute is sufficient. For example, suppose a lady needs to join a conference on
time, and she can only get the current time from her computer. If the system
time of the computer is a few minutes off, compared with the ``correct'' time,
it may be acceptable. However, if the offset amount is one hour, she will be
there either too early and have to wait for one hour or too late and the
conference may be over by that time.

In some other cases, a very high accuracy of system time is required. For
example, consider a system which handles buying and selling transactions of a
stock market.  Generally, more than one device is accessing the system.
Consequently, the system times among devices should be ``the same'' since we
need to know the order of the transactions at least. If the system is very busy
and thousands of orders are created in one second, every device should have a
system time accurate to one millisecond at least.

As we can see in the previous example, only a ``local accuracy'' is required,
which means that every device in the system needs to synchronize its system
time to the same source. If the source is in a local device, such as a local
computer, they do not
have to be synchronized to any external devices. In some other cases, a device
may need to be synchronized to a standard time. 
Coordinated Universal Time (UTC) is one kind of standard time, it is similar 
to Greenwich Mean Time (GMT) but the latter one is not a standard one. 
In practice, UTC and GMT share the same current time, but the main difference
between them is that GMT is a timezone while UTC is not~\cite{utc}. When we are
talking about time in GMT\null, we are talking about the time in that time
zone.  UTC is used more commonly, computer operating systems like Linux use UTC
to represent their system clocks. When we are talking about the time in UTC, we
are talking about the standard time. UTC is kept using atomic clocks, which can
be considered to be very accurate. UTC is widely used for time synchronization
purposes.

An easy way of synchronizing a clock is to ask someone else for the time, and
then adjust the system clock manually. We adjust the time of our
watches like that. This level of accuracy is quite sufficient if we do not need
our watches accurate to one second or better. 

Suppose we are going to design a method to synchronize computer system clocks.
The first thought might to use the method described in the last paragraph.
Assume computer $C_1$ wants to synchronize to computer $C_2$, $C_1$ can
send a message to $C_2$ then $C_2$ checks its time and replies to $C_1$. Then
$C_1$ can adjust its system time.  However, there are a lot of problems with
this simple scheme. The core one is that when $C_1$ receives the time, it only
knows when $C_2$ checked its time. $C_1$ cannot know how long it is from then
to when $C_1$ receives the message. To solve these problems, the \emph{Network
Time Protocol} (NTP) has been developed for devices to synchronize their system
times over networks. 

\section{Computers' Clocks}
\label{sec:computers_clocks}
Different computers may use different hardware methods to maintain system time.
These relate to hardware architecture and operating system design, but they
have the same fundamental idea: increasing a counter at a ``constant''
frequency, such as 1~GHz. Based on the frequency and the counter, the system
time can be calculated when necessary. However, in practice, this frequency is
not exactly the design frequency.  First of all, there may be some
manufacturing error, which makes it slightly different from what it should be.
Second, the frequency can be affected by temperature, humidity and so on. So
the system clock can be very inaccurate if it has not been synchronized for
quite a long time, and the offset from the correct time is unpredictable.

\section{Trend}
\label{sec:trend}
The performance of time synchronization is affected by the hardware
performance, the quality of network connection and the synchronization protocol
itself. 
As we have faster internet connections, and more powerful CPUs, we have a
better opportunity to more accurately set system clocks. As we have more
accurate system clocks, we can do more time sensitive activities via the internet.
As we do more of that kind of activity over the internet, or
other tasks which require highly precise time, we desire more accurate
system clocks.  We always want to make system clocks as accurate as possible
following the improvements of internet connections and CPUs. 

Currently, atomic clocks are considered the most accurate. As of 2015, the most
accurate atomic clock was accurate to $2.1\times 10^{-18}$
second~\cite{atomic_clock}.
% https://www.nature.com/articles/ncomms7896#introduction
This accuracy is far beyond the timekeeping ability of current general-purpose
computers to maintain, and beyond the current ability of networks to accurately
transmit the time from an atomic clock to a distant computer. With this fact,
we can continue to improve the performance of time synchronization as we
significantly improve hardware or network capability.

\section{Contribution}
\label{sec:contribution}
The main purpose of this thesis is to investigate the algorithms in NTP\null.
While the official documents cover most technical issues, understanding it is
in some respect complex and requires a strong background in mathematics and
engineering. This thesis explains the algorithms and makes some comments to
help people have a better understanding of NTP\null.  It will be helpful for
people who are interested in the NTP algorithms.

