% introduction.tex

% time used in daily life
% computer, even embed system keeps track of time
% different requirement of time accuracy
% why it is hard to keep time
% Note: definition of UTC
% Dr. Zhang's advise: add contribution

% why it is important
% why it is hard
% the increasing requirement of accurate time 
% computer clock structure

\chapter{Introduction}
Time is frequently use in our daily life. For example, let's consider a young
woman who works in a company. In weekdays, she may wake up at 7 am, get to
office before 9 am, have a lunch break from 11:30 to 1:30, go back home at 5
pm, have diner at 7 pm, go to sleep at 11 pm. Some of the times are based on
her personal habit, while some are required by the company she works for.
Some of them is fixed, for example, she must get to the office before 9 am,
otherwise there will be a punishment in her salary. Some of them may differ
everyday, such as the time she has diner. Sometimes people get diner not
because they are hungry, but it is the time to do it. The schedule may be
different among people, but the requirement of knowing time are the same.

There are varies ways to let people know the time, clocks and watches are
commonly used. In old days, people use huge clock to let everyone who can see
it keep track of time by frequently look at it. Even today, hand watch is
still popular no matter whether it uses quartz, which can let people get time
easily and the only requirement is to synchronize it periodically.

In modern life, computers are widely used. Most computers, including
smartphones have their internal clocks to keep track of time. For computers, 
time are widely used. A program which makes log file needs to know the time 
when an event happened which it records. Some program can change the background 
color based on time. For example, it reduce the blue color during the night
time to protect users' eyes. Smartphones can be considered as small computers 
and they become one of the main sources that people can get time from. Even
simple digital systems like parking meters or home thermostats have their
internal clocks. A parking meter needs a clock to determine when the parking
expires. People may want their heaters turned down when they are not at home
or sleeping and turned up before they get home and wake up. A home thermostat
needs a clock so that it can automatically do these for them.

In practice, different accuracies are required for different purposes. 
In some cases, if the system clock of a computer which is accurate to one
minute is enough. For example, if someone needs to join a conference on time,
and she can only get the current time from her computer. If the system time
of the computer is a few minutes off, compared with the ``correct'' time, it
may be acceptable. But if the offset amount is one hour, she will be there
either too early and have to wait for one hour or too late and the conference
may be over at that
time.

In some other cases, a very high accuracy of system time is required. For
example, there is a system handles stock market buy and sell orders. Generally,
more than one devices are there in the system. So that the system times among
devices should be ``the same'' since we need to know the order of the orders at
least. If the system is very busy and thousands orders may be created in one
second, every device should have system time accurate to millisecond at least.

As we can see in the previous example, only a ``local accuracy'' is required,
which means that every devices in the system need to synchronize its system
time to the same reference clock. If the reference clock is in an internal
device, they do not have to be synchronized to external devices. In some other
cases, a device may need to be synchronized to a standard time. 
UTC (Coordinated Universal Time) is one kind of standard time. 
(Note: explain UTC here)

An easy way of synchronizing a clock is to ask someone else for the time, and
then adjusting the system clock manually. We always adjust the time of our
watches like that. This level of  accuracy is quite enough if we do not need
our watches accurate to the second. 

What about to imply this method on synchronizing computers system clock? Assume
computer A wants to synchronize to computer B, A can send a message to B then B
checks its time and replies to A\null. Then A can adjust its system time.  
But there are a lot of problems.  The core one is that when A receives
the time, it can only know the time was when B checked its time. A cannot know
how long it is from then to when A receives the message. To solve these
problems, the network time protocol (NTP) is developed for devices to
synchronize their system times over networks. 

\section{Computers' Clocks}
\label{sec:computers_clocks}
Different computers may use different hardware methods to maintain system time.
It relates to hardware architecture and operating system design. But they have
the same fundamental idea: increasing a counter with a ``constant'' frequency,
such as 1 GHz.
Based on the frequency and the counter, the system time can be calculated
when necessary. But in practice, the frequency is not constant. First of all,
there may be some manufacturing error, which makes it slightly different from
what it should be. Second, the frequency can be affected by temperature,
humidity and so on. So the system clock can be very inaccurate if it has not
been synchronized for quite a long time, and the offset is unpredictable.

\section{Trend}
\label{sec:trend}
The performance of time synchronization is affected by the hardware
performance, the quality of network connection and the synchronization protocol
itself. 
As we have faster internet connections, and more powerful CPUs, we have a
better opportunity to more accurately set system clocks. As we have more
accurate system clocks, we can do more real-time stuff on internet. As we do
more real-time stuff on internet, we desire more accurate system clock. We
always want to make system clocks as accurate as possible after the
improvements of internet connection and CPUs. 

In real life, atomic clocks are considered the most accurate. In 2015 the most
accurate atomic clock is accurate to $2.1\times 10^{-18}$
second.~\cite{atomic_clock} 
% https://www.nature.com/articles/ncomms7896#introduction
This accuracy is far beyond the timekeeping ability of current general-purpose
computers to maintain, and beyong the current ability of networks to accurately
transmit the time from an atomic clock to a distant computer. With this fact,
we can always improve the performance of time synchronization after we
significantly improve hardware or network connection.

\section{Contribution}
\label{sec:contribution}
The main purpose of this thesis is to investigate the algorithms in NTP. While 
the official document covers all details about all technique issues, understanding
it is in some respect too complex and requires strong background about engineering.
This thesis is going to explain the algorithms and make some comment to help 
people have a better understanding of NTP and it will be helpful for people who
only interested in the algorithms.
