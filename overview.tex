% overview.tex

\chapter{Overview}
% Three kinds of devices.
NTP operates on three different kinds of devices: primary servers, secondary
servers and clients. Primary servers are directly synchronized to reference
clocks and provide service that other devices can synchronize to them.
Secondary severs are synchronized to other servers instead of reference clocks
and provide synchronization service. Clients are synchronized to other servers
but they do not provide synchronization to others.

\section{Protocol modes}%
\label{sec:protocol_modes}
NTP has three basic protocol modes: \emph{server/client}, \emph{symmetric}
and \emph{broadcast}. They are also called association modes. 
%(NOTE: I DID NOT FIND ANY DEFINITION OF ASSOCIATION!)
In the server/client variant, clients send requests to server for
synchronization, and servers handle the request and send response. We say
"clients pull synchronization from servers". The broadcast is like the
opposite mode of server/client mode, in which servers push synchronization to
clients. Symmetric mode is a combination of two server/client modes and two
broadcast modes. In symmetric variant, a peer operates as both client and
server and each peer both push and poll synchronization to and from each
other.~\cite{rfc5905}

There are two interleaved modes which we will discuss in
Section~\ref{sub:timestamps_in_ntp_packet}.

\section{Layered Network}
\label{sec:Layered_network}
The NTP network or subnet has a hierarchy architecture, which is divided into
several levels. Each level is called a stratum. All primary servers are in
stratum one, which is the highest stratum. Secondary servers are assigned the
stratum number one great than the server's stratum number which it synchronizes
to. E.g., if a secondary server synchronizes to a server in stratum 3, it is in
stratum 4. The maximum stratum number is 16.~\cite{rfc5905}
Typically, clients do not have stratum number, but we can treat them
as in a stratum by using the same rule as secondary servers.
As the stratum number increases, the accuracy of devices in stratum
decreases.

Figure~\ref{fig:layered_network} shows the basic structure of NTP subnets. The
arrows indicate the direction of synchronization. An arrow from server 3 to
server 1 indicates that server 3 communicates with server 1 and tries to
synchronize to it but server 3 may not actually synchronize to server 1.

% This figure is made by myself, may need improvement
% fig:layered_network
% figures/overview-layer.tex

\begin{figure}[htpb]
\begin{center}
\begin{tikzpicture}[scale=0.7, transform shape,
        squarednode/.style={rectangle, draw=black, very thick, minimum
        size=10mm, minimum width=20mm, align=center},
    ]
    % Nodes
    % reference clocks
    \node[squarednode]  (rc)                {Reference Clocks};
    % servers
    \node[squarednode]  (s2)      [below=of rc] {Server 2};
    \node[squarednode]  (s1)      [left=of s2]  {Server 1};
    \node[align=center] (lab1)    [left=10 mm of s1]  
                        {Stratum 1\\ (Primary Servers)};
    \node[align=center] (lab2)    [right=of s2] {\dots};

    \node[squarednode]  (s3)      [below=of s1] {Server 3};
    \node[squarednode]  (s4)      [below=of s2] {Server 4};
    \node[align=center] (lab3)    [below=12mm of lab1]  {Stratum 2};
    \node[align=center] (lab4)    [right=of s4] {\dots};

    \node[align=center, rotate=90] (lab5)    [below=10mm of lab3] {\dots};

    \node[align=center] (lab6)    [below=17mm of lab3]  {Stratum n};
    \node[squarednode]  (client)  [below=13mm of s3] {Client};
    
    % synchronizations
    \draw[-latex, thick] (s1) -- (rc);
    \draw[-latex, thick] (s2) -- (rc);
    \draw[-latex, thick] (s3) -- (s1);
    \draw[-latex, thick] (s3) -- (s2);
    \draw[-latex, thick] (s4) -- (s2);
    \draw[-latex, thick] (client) -- (s2);
    \draw[-latex, thick] (client) -- (s3);
    \draw[-latex, thick] (client) -- (s4);

\end{tikzpicture}
\end{center}
\caption{NTP Layered Network}
\label{fig:layered_network}
\end{figure}




% Feb 20 2019, Dr. Diamond gave some advise about this paragraph
As people want their devices to be accurate, it is better to synchronize to
primary servers. But this can make primary servers very busy. Based on the rule
of engagement for NTP users, ``clients should avoid using primary servers
whenever possible''. Even though, some busy primary servers have more than 700
clients. 
% http://support.ntp.org/bin/view/Servers/RulesOfEngagement
It is said that the synchronizations provided by secondary servers in stratum 2
are accurate enough for most end-users. And clients are suggested to use three
to five servers. In practice, there are some pools of secondary servers.
Clients can assign a pool as a server which they want to synchronize to. When
the pool receive the connection request, it uses DNS round robin to select a
random server from the pool. Then client actually connects to the selected
server.
% http://support.ntp.org/bin/view/Servers/NTPPoolServers

\section{Processes}
\label{sec:processes}
% Architecture of client processes.
As Figure~\ref{fig:architecture_overview} shows, operations of an NTP client
can be separated into the following processes:
\begin{itemize}
    \item Peer/poll Processes\\
        This part deal with communications between client and servers. It has
        poll processes and peer processes. Poll processes send requests to
        servers. Peer processes keep track of a set of statistics for every
        server.
    \item System Process\\
        System process has algorithms to do some filter and correction work to
        the statistics of all peers, then pass them to clock discipline
        process.
    \item Clock Discipline Process\\
        Clock discipline process is like a lowpass filter to smooth the data.
        This process also maintains the poll intervals.
        % red book p. 40
    \item Clock Adjust Process\\
        This part adjusts the system clock to make it continuous and monotonic
        approximately. 
        % red book p. 40
\end{itemize}
The structure of NTP is like a loop. It requests synchronization to servers, after
gets response, it deals with it then adjust system time. After that, there are some
feedback used to control poll processes.

In later chapters, we will discuss peer/poll processes and system process
separately, and discuss clock discipline process and clock adjust process
together. Note that, in NTP official documents, ``server'' and ``peer'' are
used interchangeably.

% This figure is made based on the figure from
% https://www.eecis.udel.edu/~mills/ntp/html/warp.html
% fig:architecture_overview
% overview-arch.tex

%\begin{figure}[htpb]
%    \centering
%    \includegraphics[width=0.8\linewidth]{../figures.old/fig_3_1.png}
%    \caption{NTP architecture overview}
%    \label{fig:arch_overview}
%\end{figure}


\begin{figure}[htpb]
\begin{center}
\begin{tikzpicture}[scale=0.7, transform shape,
        squarednode/.style={rectangle, draw=black, very thick, minimum
        size=15mm, minimum width=30mm, align=center},
        squarednode1/.style 2 args={draw=black, very thick, minimum
        size=15mm,fit={(#1.north) (#2.south)}, text width=, inner sep=2mm, text
        centered, align=center},
        circlenode/.style={circle, draw=black, very thick, minimum size=15mm},
        border/.style={-, draw=black, very thick, dashed},
    ]
    % Nodes
    % remote servers, peer/poll process
    \node[squarednode]  (s1)                {Server 1};
    \node[squarednode]  (s2)  [below=of s1] {Server 2};
    \node[squarednode]  (s3)  [below=of s2] {Server 3};
    \node[align=center](lab1)[below=of s3] {Remote\\ servers};

    \node[squarednode]  (p1)  [right=of s1] {Peer/poll 1};
    \node[squarednode]  (p2)  [right=of s2] {Peer/poll 2};
    \node[squarednode]  (p3)  [right=of s3] {Peer/poll 3};
    \node[align=center](lab2)[below=of p3] {Peer/Poll\\ processes};

    \draw[border] ($(p1.north west) + (-0.4, 0.2)$) -| ($(lab2.south east) +
    (0.8, -0.2)$);
    \draw[border] ($(p1.north west) + (-0.4, 0.2)$) |- ($(lab2.south east) +
    (0.8, -0.2)$);

    % system process
    \node[squarednode]  (sa1) [right=of p2]
    {Selection\\ and\\ clustering\\ algorithms};
    \node[squarednode] (sa2) [right=of sa1] {Combine\\ algorithms};
    \node[align=center](lab3)[above=5mm of sa2] {System\\ process};

    \draw[border] ($(lab3.north east) + (1.0, 0.2)$) -| ($(sa1.south west) +
    (-0.4, -0.8)$);
    \draw[border] ($(lab3.north east) + (1.0, 0.2)$) |- ($(sa1.south west) +
    (-0.4, -0.8)$);

    % clock discipline process
    \node[squarednode] (lf) [right=of sa2] {Loop filter};
    \node[align=center](lab4)[above=5mm of lf] {Clock discipline\\ process};
    \draw[border] ($(lab4.north east) + (0.3, 0.2)$) -| ($(lf.south west) +
    (-0.4, -0.8)$);
    \draw[border] ($(lab4.north east) + (0.3, 0.2)$) |- ($(lf.south west) +
    (-0.4, -0.8)$);

    % clock adjust process
    \node[circlenode] (vfo) [below=25mm of lf] {VFO};
    \node[align=center](lab5)[below=5mm of vfo] {Clock adjust\\ process};
    \draw[border] ($(vfo.north east) + (1.0, 0.5)$) |- ($(lab5.south west) +
    (-0.2, -0.2)$);
    \draw[border] ($(vfo.north east) + (1.0, 0.5)$) -| ($(lab5.south west) +
    (-0.2, -0.2)$);
    
    % Lines
    % server-peer/poll process
    \foreach \x/\y in {s1/p1, s2/p2, s3/p3} {
        %\draw[-] (\x) -- (\y);
        \draw[-latex, thick] ($(\x.north east)!0.5!(\x.east)$) --
        ($(\y.north west)!0.5!(\y.west)$); 
        \draw[latex-, thick] ($(\x.south east)!0.5!(\x.east)$) --
        ($(\y.south west)!0.5!(\y.west)$); 
    }
    % peer/poll process
    \draw[-latex, thick] (p2.north) -- (p1.south);
    \draw[-latex, thick] (p3.north) -- (p2.south);
    % peer/poll process-system process
    \draw[-latex, thick] (p1.east) -- (sa1.north west);
    \draw[-latex, thick] (p2.east) -- (sa1.west);
    \draw[-latex, thick] (p3.east) -- (sa1.south west);
    % system process
    \draw[latex-, thick] (sa2.west) -- (sa1.east);
    % system process-clock discipline process
    \draw[-latex, thick] (sa2.east) -- (lf.west);
    % clock discipline process-clock adjust process
    \draw[-latex, thick] (lf.east) -- ($(lf.east) + (1,0)$) |- (vfo.east);
    % clock adjust process-poll process
    \draw[-latex, thick] (vfo.west) -| (p3.south);
    

\end{tikzpicture}
\end{center}
\caption{NTP architecture overview}
\label{fig:architecture_overview}
\end{figure}




% (What is the stuff I want to add here when I was taking shower on Fer 5?)

