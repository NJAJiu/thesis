% overview.tex

\chapter{Overview}
% Three kinds of devices.
NTP operates on three different kinds of devices: primary servers, secondary
servers and clients. Primary servers are directly synchronized to reference
clocks and provide service that other devices can synchronize to them.
Secondary severs are synchronized to other servers instead of reference clocks
and provide synchronization service. Clients are synchronized to other servers
but they do not provide synchronization to others.

\section{Layered Network}
\label{sec:Layered_network}
The NTP network or subnet has a hierarchy architecture, which is divided into
several levels. Each level is called a stratum. All primary servers are in
stratum one, which is the highest stratum. Secondary servers are assigned the
stratum number one great than the server's stratum number which it synchronizes
to. E.g., if a secondary server synchronizes to a server in stratum 3, it is in
stratum 4. The maximum stratum number is 16. (RFC5905,p.17)
Typically, clients do not have stratum number, but we can treat they
as in a stratum by using the same rule as secondary servers.
As the stratum number increases, the accuracy of devices in stratum
decreases.

Figure~\ref{fig:layered_network} shows the basic structure of NTP subnets. The
arrows indicate the direction of synchronization. An arrow from server 3 to
server 1 indicates that server 3 communicates with server 1 and tries to
synchronize to it but server 3 may not actually synchronize to server 1.

% This figure is made by myself, may need improvement
% fig:layered_network
\input{figures/overview-layer}

As people want their devices to be accurate, it is better to synchronize to
primary servers. But this can make primary servers very busy. Based on the rule
of engagement for NTP users, ``clients should avoid using primary servers
whenever possible''. Even though, some busy primary servers have more than 700
clients. (Note: Clients?)
% http://support.ntp.org/bin/view/Servers/RulesOfEngagement
It is said that the synchronizations provided by secondary servers in stratum 2
are accurate enough for most end-users. And clients are suggested to use three
to five servers. In practice, there are some pools of secondary servers.
Clients can assign a pool as a server which they want to synchronize to. When
the pool receive the connection request, it uses DNS round robin to select a
random server from the pool. Then client actually connects to the selected
server.
% http://support.ntp.org/bin/view/Servers/NTPPoolServers

\section{Processes}
\label{sec:processes}
% Architecture of client processes.
As figure~\ref{fig:architecture_overview} shows, operations of an NTP client
can be separated into the following processes:
\begin{itemize}
    \item Peer/poll Processes\\
        This part deal with communications between client and servers. It has
        poll processes and peer processes. Poll processes send requests to
        servers. Peer processes keep track of a set of statistics for every
        server (peer?). 
    \item System Process\\
        System process has algorithms to do some filter and correction work to
        the statistics of all peers, then pass them to clock discipline
        process.
    \item Clock Discipline Process\\
        Clock discipline process is like a lowpass filter to smooth the data.
        This process also maintains the poll intervals.
        % red book p. 40
    \item Clock Adjust Process\\
        This part adjusts the system clock to make it continuous and monotonic
        approximately. (Note: feedback?)
        % red book p. 40
\end{itemize}

% This figure is made based on the figure from
% https://www.eecis.udel.edu/~mills/ntp/html/warp.html
% fig:architecture_overview
\input{figures/overview-arch}

(What is the stuff I want to add here when I was taking shower on Fer 5?)

