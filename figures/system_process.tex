% figurs/system_process.tex

\begin{figure}[htpb]
    \begin{center}
    \begin{tikzpicture}[scale=0.7, transform shape,
            squarednode/.style={rectangle, draw=black, very thick, minimum
            size=15mm, minimum width=30mm, align=center},
            squarednode1/.style 2 args={draw=black, very thick, minimum
            size=15mm,fit={(#1.north) (#2.south)}, text width=, inner sep=2mm, text
            centered, align=center},
            circlenode/.style={circle, draw=black, very thick, minimum size=15mm},
            border/.style={-, draw=black, very thick, dashed},
        ]
    % Nodes

    % system process
    \node[squarednode] (sa1) {Selection\\ algorithm};
    \node[squarednode] (sa2) [right=of sa1] {Clustering\\ algorithm};
    \node[squarednode] (sa3) [right=of sa2] {Combining\\ Algorithms};
    \node[align=center](lab3)[above=5mm of sa2] {System\\ Process};
    \draw[border] ($(lab3.north east) + (1.0, 0.2)$) 
        -| ($(sa1.south west) + (-0.4, -0.8)$) 
        -- ($(sa3.south east) + (0.4, -0.8)$) 
        |- ($(lab3.north east) + (1.0, 0.2)$);

    % peer
    \node[squarednode] (p) [left=of sa1] {Peer\\ statistics};

    % clock discipline
    \node[squarednode] (cd) [right=of sa3] {Clock discipline\\ process};
    % Lines

    % system process
    \draw[-latex, thick] (sa1.east) -- (sa2.west);
    \draw[-latex, thick] (sa2.east) -- (sa3.west);
    \draw[-latex, thick] (p.east) -- (sa1.west);
    \draw[-latex, thick] (sa3.east) -- (cd.west); 
        
    
\end{tikzpicture}
\end{center}
\caption{System process overview}
\label{fig:system_process}
\end{figure}
    
    
    
