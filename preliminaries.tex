%---------------------------------------------------------
% Preliminaries: Set up your own details in this file!
%----------------------------------------------------------

% Don't forget to remove the ()s in ALL of these "personalization" lines.

\title{An investigation of Network Time Protocol}
\author{Hanjie Li}
\dept{Computer Science}   % E.g., Physics, Computer Science,
\deptOrSchool{School}         % Pick one, remove the rest
\degree{Computer Science}  % E.g., Science, Arts, ...

\submissionMonth{March}	    % OR WHATEVER MONTH YOU ACTUALLY SUBMIT IN
\submissionYear{2019}
\copyrightYear{2019}		    % Probably the same as submissionYear.

% Use a "~" after the "r." of "Dr." so that TeX doesn't think you have
% ended a sentence (at which point it gives extra space).
\supervisor{Dr.~James Diamond}

% Remove the '%' from the next line and fill in the name if desired.
\cosupervisor{Dr.~Haiyi Zhang}

\headOrDirector{Dr.~Darcy Benoit}
% If the head or director is an ``acting'' head or director, uncomment
% the next line (i.e., delete the '%'):
% \justActing

% You will have to ask around to find out the name of the person
% to put here... it changes from year to year.
\honoursCommittee{Dr.~Joseph Hayes}
% https://research.acadiau.ca/Undergraduate_Student_Honours_Research.html

%-------------------------------------------------------------------------

% This outputs the title page, the approval page and the copyright page.
\firstThreePages

%-------------------------------------------------------------------------
% Now write your acknowledgements (if any).
% If you wish to acknowledge no-one, delete or comment-out the
% next few lines.

\Acknowledgments
I would first want to thank my supervisor Dr.~James~Diamond for his correction
of this thesis. I thank him for spending a large amount of time helping me to
improve my writing skills. I also want to thank him for what he taught me
during my studies at Acadia University and his hard work for all the students.

I would also like to thank my wonderful wife Fan Zhang who always supported me,
even though the distance between us was more than ten thousand kilometers for
most of the time in the past three years.

I also want to thank my co-supervisor Dr.~Haiyi~Zhang who gave me a lot of
suggestions to make this thesis easy to understand.

%-------------------------------------------------------------------------

% This outputs the table of contents, lists of figures, tables, ...

\tocAndSuch

%-------------------------------------------------------------------------

\prefacesection{Abstract}

Time is widely used in life: most people want to accurately know the time. If
we have a mechanical watch, maybe it is sufficient to make it accurate to one
second. However, on computers, especially on computers with high speed CPUs and
networks, people want their system time to be more accurate.  Time is sensitive
for some programs, such as logging programs or a program related to stock
market transactions. The Network Time Protocol is designed to synchronize
computers' system times. The basic idea is to let a computer communicate with
servers whose system times are synchronized, then based on the response, the
computer can change its system time. There are some atomic clocks which are
extremely accurate. Some servers can synchronize to them, and more servers can
synchronize to synchronized servers. The servers become a network.  When we
want to synchronize our computer, we can try to synchronize to servers in the
network.  The difficulty is we do not know how long a request packet takes to
travel from us to servers, how long it takes for a server to deal with the
request and how long a response packet takes to travel back to us. The Network
Time Protocol is designed to overcome, as much as possible, this difficulty.
This thesis investigates how the Network Time Protocol operates and explains
the complex algorithms employed.
%-------------------------------------------------------------------------

% Don't mess with this line!
\afterpreface
