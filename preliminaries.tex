%---------------------------------------------------------
% Preliminaries: Set up your own details in this file!
%----------------------------------------------------------

% Don't forget to remove the ()s in ALL of these "personalization" lines.

\title{An investigation of Network Time Protocol}
\author{Hanjie Li}
\dept{Computer Science}   % E.g., Physics, Computer Science,
\deptOrSchool{School}         % Pick one, remove the rest
\degree{Computer Science}  % E.g., Science, Arts, ...

\submissionMonth{March}	    % OR WHATEVER MONTH YOU ACTUALLY SUBMIT IN
\submissionYear{2019}
\copyrightYear{2019}		    % Probably the same as submissionYear.

% Use a "~" after the "r." of "Dr." so that TeX doesn't think you have
% ended a sentence (at which point it gives extra space).
\supervisor{Dr.~James Diamond}

% Remove the '%' from the next line and fill in the name if desired.
\cosupervisor{Dr.~Haiyi Zhang}

\headOrDirector{Dr.~Darcy Benoit}
% If the head or director is an ``acting'' head or director, uncomment
% the next line (i.e., delete the '%'):
% \justActing

% You will have to ask around to find out the name of the person
% to put here... it changes from year to year.
\honoursCommittee{Dr.~Joseph Hayes}
% https://research.acadiau.ca/Undergraduate_Student_Honours_Research.html

%-------------------------------------------------------------------------

% This outputs the title page, the approval page and the copyright page.
\firstThreePages

%-------------------------------------------------------------------------
% Now write your acknowledgements (if any).
% If you wish to acknowledge no-one, delete or comment-out the
% next few lines.

\Acknowledgments

Place any acknowledgments you might want to make here.

\noindent
Don't forget to be formal and professional.

%-------------------------------------------------------------------------

% This outputs the table of contents, lists of figures, tables, ...

\tocAndSuch

%-------------------------------------------------------------------------

\prefacesection{Abstract}

Time is widely used in life. All people wants to keep their times accurate. If
we have a mechanical watch, maybe it is sufficient to make it accurate to one
second. However, on computers, especially on computers with high speed CPUs and
networks, people want their system time to be more accurate.  Time is sensitive
for some programs, such as logging program or a program relates to stock
market. Network Time Protocol is designed to synchronize computers' system
time. The basic idea is to let a computer communicate with servers whose system
time are synchronized, then based on the response, the computer can change its
system time. There are some atomic clocks which are extremely accurate. Some
servers can synchronize to them, and more servers can synchronize to
synchronized servers. The servers become a network.  When we want to
synchronize our computer, we can try to synchronize to servers in the network.
The difficulty is we do not know how long a request packet takes to travel from
us to servers, how long it takes for a server to deal with the request and how
long a response packet takes to travel back to us.  This thesis is going to
investigate algorithms of Network Time Protocol to see how to solve these
problems.

%-------------------------------------------------------------------------

% Don't mess with this line!
\afterpreface
