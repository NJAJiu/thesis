% background.tex

\chapter{Background}

\section{History and current situation}%
\label{sec:history_and_current_situation}
The basic idea of NTP was developed in the early 1980s. NTP version 0 was
implemented in 1985, and it is documented in RFC-958~\cite{rfc958}. 
% But the actual implementation did a little bit more. 
The NTP packet header and the calculations of offset and delay (as will be
discussed later) from that NTP version are still being used today. At that
time, the nominal accuracy could be in the low tens of milliseconds on an
Ethernet network and better than 100 milliseconds on paths spanning the
Atlantic Ocean.

NTP version 1 was documented in RFC-1059 one year later. It contains a
specification of the protocol and algorithms.

NTP version 2 specification is documented in RFC-1119~\cite{rfc1119} in 1989.
This version
includes new protocols for use in managing NTP servers and clients and a
cryptographic authentication scheme.
At about the same time, another time synchronization protocol was presented,
which is Digital Time Synchronization Service (DTSS)~\cite{DTSS}.

NTP version 3 specification appeared in 1992, following
RFC-1305~\cite{rfc1305}. This version combined the good ideas from both NTP and
DTSS\null.

After that, Simple Network Time Protocol version 4 was documented in
RFC-2030~\cite{rfc2030}
in 1996. SNTP is used if users do not need the ultimate performance of NTP or
it is not otherwise justified.

The latest version of NTP follows RFC-5905~\cite{rfc5905}, which is NTP version
4 and presented in 2010.
% history.pdf

\section{Reference clocks}%
\label{sec:reference_clocks}
``A reference clock is some device or machinery that spits out the current
time''~\cite{reference_clock}.
% http://www.ntp.org/ntpfaq/NTP-s-algo.htm#Q-REFCLK
The most important aspect of a reference clock is its accuracy; it must be
synchronized to some time standard (usually UTC). If we want to synchronize a
device to UTC, it requires reference clocks as sources which are themselves
synchronized to UTC\null. Some of them are
available for many government dissemination services, includes:~\cite{redbook}
\begin{itemize}
    \item Global Positioning System (GPS) and Long-Range navigation (LORAN-C)
        systems
    \item WWV/H and WWVB radio time/frequency stations
    \item U.S. Naval Observatory (USNO) and National Institutes of Science and
        Technology (NIST\null, formerly the National Bureau of Standards [NBS])
        telephone modem services in the United States
    \item Similar systems and services in other countries
\end{itemize}
Generally, we want all devices synchronized to UTC as well as reference clocks.
But in some case, we may just want all devices in a local area network
synchronized to a certain ``standard'' time such as the system time of a
certain device among them. In this case, no UTC reference clock is in the
subnet. However, we can consider the device to which others are synchronized as
a reference clock. 

\section{Adjusting system time}%
\label{sec:adjusting_system_time}
As mentioned in Section~\ref{sec:computers_clocks}, every computer should have
a constant frequency to increase the number in a counter which is used to
calculate system time. However, in practice, the frequency may be nether
accurate nor consistent. In this case, even if at some time we synchronized the
system time to UTC or a reference clock, it will be inaccurate later. This fact
makes it necessary to synchronize a device frequently.

When system clock is adjusted by NTP\null, both the time and the frequency
will be adjusted. For Unix systems, NTP uses \verb|ntp_adjtime()| or
\verb|adjtime()| system call to adjust system time when the offset is small.
The benefit of using them is that they will synchronize the system time step by
step over a period of time by adjusting a small amount in one step and finally
make it synchronized. In this case, any running program which is sensitive to
time, such as a logging program, will only be slightly impacted. If we want to
change system time back for some amount, it will not make later things
``happened'' earlier in the log. If we want to change system time forwardly,
there will not be a gap in the time line, where nothing happened during that
time. However, if the offset it big, NTP will also change the system time to
another value.

% \section{Concepts}%
% \label{sec:concepts}

% (Note: should this part be here or somewhere else or no need?)


