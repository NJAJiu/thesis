% peer_poll.tex
\chapter{Peer/Poll Processes}
\label{cha:peer/poll}

\section{Concepts}%
\label{sec:peer/poll_concepts}
\begin{enumerate}
    \item Offset $\theta$\\
        The maximum-likelihood time offset of the server clock relative to the
        system clock. This concept describes how different the client's time
        and the server's time are.
        % $$\theta_{AB} = \frac{1}{2}\left[\left(T_{i-2} - T_{i-3}\right) +
        % \left(T_{i-1} - T_i\right)\right]$$
    \item Delay $\delta$\\
        The round-trip delay between the client and server. It is equal to the
        time that a massage travels from the client to the server and the time
        it comes back.
        % $$\delta_{AB} = \left(T_{i} - T_{i-3}\right) - \left(T_{i-1} -
        % T_{i-2}\right)$$
    \item Dispersion $\varepsilon$\\
        The maximum error inherent in the measurement.
        Another definition: Potential clock offset error due to the maximum
        uncorrected system clock frequency error.
        These definitions are in some respect confused, we will discuss it with
        more detail in section~\ref{sec:peer_processes}.
        % from website https://www.eecis.udel.edu/~mills/ntp/html/stats.html
\end{enumerate}

\section{Overview}%
\label{sec:overview}
Figure~\ref{fig:peer_poll_processes} shows the basic architecture of peer/poll
processes. The communication between client and server uses on-wire protocol,
which can provide sample statistics ($\theta_0, \delta_0, t_0$ in the figure).
The recent eight samples are stored in a shift register, with one more sample
statistic added ($\varepsilon$).  Then peer process applies clock filter
algorithm, which generates peer statistics ($\theta, \delta, \varepsilon$ in
the figure), and passes them to system process.


% figures/peer_poll_figure.tex

\begin{figure}[htpb]
\begin{center}
\begin{tikzpicture}[scale=0.7, transform shape,
        squarednode/.style={rectangle, draw=black, very thick, minimum
        size=10mm, minimum width=20mm, align=center},
        border/.style={draw=black, dashed, very thick},
    ]
    % Nodes
    % server
    \node[squarednode]  (s)                {Server};

    % on-wire protocol
    \node[align=center] (onwire) at ($(s.east) + (2.5, 2)$) {On-wire Protocol};
    \draw[border] ($(onwire.north west) + (-0.2, 0.2)$) -|
            ($(onwire.north east) + (0.2, -5)$) -|
            ($(onwire.north west) + (-0.2, 0.2)$);
    % poll process
    \node[squarednode] (poll) at ($(onwire) + (0, -6)$) {Poll Process};
    \draw[-latex, ultra thick] (poll) |- ($(s.south) + (0, -0.5)$) -- (s);

    % shift registers
    \node[squarednode] (sr) at ($(onwire) + (4.5,-2)$) 
            {Shift Registers\\
            $(\theta_1, \delta_1, \varepsilon_1, t_1)$\\
            $(\theta_2, \delta_2, \varepsilon_2, t_2)$\\
            $(\theta_3, \delta_3, \varepsilon_3, t_3)$\\
            $(\theta_4, \delta_4, \varepsilon_4, t_4)$\\
            $(\theta_5, \delta_5, \varepsilon_5, t_5)$\\
            $(\theta_6, \delta_6, \varepsilon_6, t_6)$\\
            $(\theta_7, \delta_7, \varepsilon_7, t_7)$\\
            $(\theta_8, \delta_8, \varepsilon_8, t_8)$
            };
    \draw[-latex, ultra thick] (s) -- (sr) node[midway, above] 
            {$(\theta_0, \delta_0, t_0)$};
    % clock filter algorithm
    \node[squarednode] (cf) [right=of sr] {Clock Filter\\ algorithm};
    \draw[-latex, ultra thick] (sr) -- (cf);

    % peer process
    \node[align=center] (peer) [below=2mm of sr.south east] {Peer Process};
    \draw[border] ($(sr.north west) + (-0.5, 0.2)$) -- 
            ($(sr.south west) + (-0.5, -1.2)$)
            -|
            ($(cf.east) + (0.2, 0)$) |-
            ($(sr.north west) + (-0.5, 0.2)$);


    % system process
    \node[squarednode] (sp) [right=2.5 of cf] {System process};
    \draw[-latex, ultra thick] (cf) -- (sp) node[midway, above] 
            {$(\theta, \delta, \varepsilon, \varphi)$};
    
    % synchronizations
    % \draw[-latex, ultra thick] (s1) -- (rc);

\end{tikzpicture}
\end{center}
\caption{Peer/Poll processes}
\label{fig:peer_poll_processes}
\end{figure}



% fig:peer_poll_processes

The purpose of peer processes is to communicate with servers and to keep
statistics of the communications which can be used to determine how reliable
the servers are. And for poll processes, their job is to manage the
communications to get sufficient data without making the servers too busy.

\section{On wire protocol}%
\label{sec:on_wire_protocol}
In NTP subnet, a client can send request to server and the server sends back
response after it got the request. The whole round of the communication can
generate a sample. For every sample, the client wants to know the difference
between the server's system clock and its, aka offset.
But it may take a while for the server to give the response. And it is not
predictable, as well as the network delay between the two devices.
In NTP subnet, clients and servers communicate under on-wire protocol. It
provides a smart way to calculate offset and delay for every sample. 

\subsection{Statistics calculation}%
\label{sub:statistics_calculation}
Figure~\ref{fig:on_wire1} represents the time stamps and communications between
a client and a server. The vertical dash line indicates the same time. The
arrow indicates the communication packages sent from one side and received at
the other side. The notation $t_i$ stands for the time stamp read from the
system clock of one device. 

% figures/on_wire1.tex

\begin{figure}[htpb]
\begin{center}
\begin{tikzpicture}[scale=0.7, transform shape,
        textnode/.style={rectangle, very thick, minimum
        size=10mm, minimum width=20mm, align=center},
        border/.style={draw=black, dashed, very thick},
        arrow/.style={-latex, ultra thick},
    ]
    % server
    \node[textnode]  (s)                {Server};
    \draw[arrow]     (s) -- ($(s) + (10,0)$);

    % client
    \node[textnode]  (c)    [below=20mm of s]            {Client};
    \draw[arrow]     (c) -- ($(c) + (10,0)$);

    % communications
    \draw[arrow]     ($(c) + (5, 0)$) -- ($(s) + (6, 0)$);
    \draw[arrow]     ($(s) + (8, 0)$) -- ($(c) + (9, 0)$);

    % vertical line
    \draw[border]    ($(s) + (3, 1)$) -- ($(c) + (3, -1)$);

    % labels
    \node[textnode] at ($(c) + (2.5, -0.5)$)   {$T_0$};
    \node[textnode] at ($(s) + (2.5, 0.5)$)    {$T_1$};
    \node[textnode] at ($(c) + (5, -0.5)$)   {$T_2$};
    \node[textnode] at ($(s) + (6, 0.5)$)    {$T_3$};
    \node[textnode] at ($(s) + (8, 0.5)$)    {$T_4$};
    \node[textnode] at ($(c) + (9, -0.5)$)   {$T_5$};

\end{tikzpicture}
\end{center}
\caption{Time stamps}
\label{fig:on_wire1}
\end{figure}



% fig:on_wire1

If we have $t_1$ and $t_0$, which are time stamps read from both devices at the
same time, then we can get the offset:
$$\theta = t_1 - t_0$$
But we cannot do this since we cannot determine whether two time stamps were
token at the same time. The on-wire protocol actually calculates the offset in
an alternative way:
$$\theta = \frac{1}{2}\left[(t_3 - t_2) + (t_4 - t_5)\right]$$

\section{Peer processes}%
\label{sec:peer_processes}

\subsection{Clock filter algorithm}%
\label{sub:clock_filter_algorithm}

\subsection{Huff-'n-puff filter}%
\label{sub:huff_n_puff_filter}

\section{Poll processes}%
\label{sec:poll_processes}


